\documentclass[10pt]{article}
\setlength{\textwidth}{6.3in}
\setlength{\textheight}{9in}
\setlength{\oddsidemargin}{0in}
\setlength{\evensidemargin}{0in}
\setlength{\topmargin}{-.5in}
%\parindent=0in
\linespread{1.3}
\usepackage{ mathrsfs }
\usepackage{amsthm}
\usepackage{ amssymb }
\usepackage{graphicx}
\newtheorem{theorem}{Theorem}[section]
\newtheorem{lemma}[]{Lemma}
\newtheorem{definition}[]{Definition}

\usepackage{amsmath}
\usepackage{amsfonts}
\usepackage{fancyhdr}
\usepackage{nicematrix}
\usepackage{mathtools}

\pagestyle{fancy}
\headheight = 14.5pt
\lhead{Functional Analysis HW1, Thomas Zeng }
\rhead{Math 331, Spring 2023}
\cfoot{\thepage}


\begin{document}

\section*{1}

Given that $f$ is convex and differentiable on some neighborhood $V_\epsilon(a),$ then there exists some $u$ s.t. $0<\delta<u$ implies that
\[\frac{1}{2\delta}\int_{a-\delta}^{a+\delta}f(x) > f(a).\]
\begin{proof}
    % We consider three cases.

    % \noindent
    % \textbf{Case 1:} ($f'(a)\ge 0$). 
    % WLOG, assume that $f$ is increasing on $V_\epsilon(a)$ i.e. there is no $x\in V_\epsilon(a)$ s.t. $f'(x)=0$ (if this is not the case, just set $\epsilon = |a-x|$).
    We first claim that for all $0<j<\epsilon$ that % put into its own lemma at begining
    \begin{equation} \label{eq:T}
        f(a+j)-f(a)>f(a)-f(a-j).
    \end{equation}
    % To prove this, we use contradiction.  Namely, assume this is not true for some fixed $j$. Therefore
    % \begin{equation} \label{eq:c}
    %     f(a+j)-f(a)\le f(a)-f(a-j).
    % \end{equation}
    By the Mean Value Theorem (as $f$ is differentiable and thus also continuous), it follows that there exists points $c\in(a-j,a)$ and $c'\in(a,a+j)$ with 
    \[f'(c) = \frac{f(a)-f(a-j)}{j} \quad\text{and}\quad f'(c') =\frac{f(a+j)-f(a)}{j}.\]
    As $c'>c,$ by definition of convexity, we have $f'(c')> f'(c)$. Therefore, we have
    \begin{alignat*}{2}
        % f'(c') &> f'(c)\\
        \frac{f(a+j)-f(a)}{j}&>\frac{f(a)-f(a-j)}{j} \qquad&& \text{by above}\\
        f(a+j)-f(a) &> f(a)-f(a-j) &&\text{as $j>0,$}
    \end{alignat*}
    and hence proving \eqref{eq:T}.

    Now for any $0<\delta<\epsilon,$ we therefore have that
    \begin{alignat*}{2}
        \int_{(0,\delta)}  f(a+j) - f(a) dj &> \int_{(0,\delta)}  f(a) - f(a-j) dj \qquad&&\text{by integral properties and \eqref{eq:T}}\\
        \int_{0}^{\delta}  f(a+j) - f(a) &> \int_{0}^{\delta}  f(a) - f(a-j) &&\text{as $\int_{(a,b)}f=\int_{[a,b]}f$}\\
        \int_{a}^{a+\delta} f(x) - f(a) dx&> \int_{a-\delta}^{a} f(a) - f(x) dx&&\text{change of variables}\\
        \int_{a}^{a+\delta} f(x) + \int_{a-\delta}^{a} f(x) &>  \int_{a}^{a+\delta} f(a) + \int_{a-\delta}^{a} f(a)\\
        \int_{a-\delta}^{a+\delta}f(x) &> \delta f(a) + \delta f(a)\\
        \frac{1}{2\delta}\int_{a-\delta}^{a+\delta}f(x) &> f(a).
    \end{alignat*}
    Thus proving our claim as desired.
    % \noindent
    % \textbf{Case 2:} ($f'(a)<0$). This case is symetric to the first case. Just negate the signs in \eqref{eq:T} (and throughout rest of proof).
    % \noindent
    % \textbf{Case 3:} ($f'(a)=0$). We claim that for all $0<j<\epsilon$ that both
    % \[f(a-j)>f(a) \quad\text{and}\quad f(a+j)>f(a).\]
    % This result again follows simply from contradiction with the Mean Value Theorem. Specifically, if \[f(a-j)\le f(a),\] then there exists $c \in (a-j, a)$ with
    % \[f'(c)= \frac{f(a) - f(a-j)}{j} \ge 0 = f(a) \quad\text{but}\quad c<a,\]
    % which contradicts the convexity assumption. A similar proof works for $f(a+j).$

    % It thus follows for any $0<\delta<\epsilon$ that
    % \[\int_{(0,\delta)}f(a-j)dj >\int_{(0,\delta)}f(a)dj \quad\text{and}\quad \int_{(0,\delta)}f(a+j)dj>\int_{(0,\delta)}f(a)dj.\]
    % Combing these two inequalities, we thus have:
    % \begin{alignat*}{2}
    %     \int_{(0,\delta)}f(a-j)dj + \int_{(0,\delta)}f(a+j)dj &> \int_{(0,\delta)}f(a)dj + \int_{(0,\delta)}f(a)dj\\
    %     \int_{a-\delta}^{a+\delta}f(x)dx &> \int_{a-\delta}^{a+\delta} f(a)dx\\
    %     \frac{1}{2\delta}\int_{a-\delta}^{a+\delta}f(x) &> f(a).
    %  \end{alignat*}
    %  as desired.
\end{proof}
\newpage

\section*{2}
The solution
\begin{equation} \label{eq:def}
    u(t,x) \coloneqq \sum_{n=1}^{\infty}c_ne^{-(\pi n)^2t}\sin(n\pi x)
\end{equation}
is continuous and solves the heat equation if $\sum_{n=1}^{\infty}|c_n|<\infty$.\\

\noindent
\textbf{We first show that $u$ is continuous.}
\begin{proof}
    As  $\sum_{n=1}^{\infty}|c_n|<\infty,$ therefore, there exists some $M\in\mathbb{R}$ s.t.  $\sum_{n=1}^{\infty}|c_n|<m,$ i.e. $\sum_{n=1}^{\infty}|c_n|$ is bounded. Therefore, as the partial sums of this series is monotone and bounded, it follows by Monotone Convergence Theorem that $\sum_{n=1}^{\infty}|c_n|$ converges.

    We now define the sequence of functions $(f_n)$ with
    \[f_n = c_ne^{-(\pi n)^2t}\sin(n\pi x).\]
    Each $f_n$ is continuous by Algebraic Continuity Theorem. 
    
    We furthermore claim that $\sum f_n \stackrel{unif}{\to} u.$ By definition of $u$ as in \eqref{eq:def}, given that $\sum f_n$ converges, it must converge to $u$. It thus suffices to show that $\sum f_n$ is uniform convergent. Specifically, for arbitrary $n\in\mathbb{N}$ and arbitrary $x,t$ we have that
    \begin{alignat*}{2}
        |f_n(x,t)| &= \left |c_ne^{-(\pi n)^2t}\sin(n\pi x)\right | \qquad&&\text{by def. of $f_n$}\\
        &\le |c_ne^{-(\pi n)^2t}| \qquad&&\text{as $-1\le \sin x \le 1$ for all $x$}\\
        &\le  |c_n| &&\text{as $0<e^{-x}\le 1$ for $x\ge0$}\\
    \end{alignat*}
    Therefore define the sequence $(M_k)$ where $M_k = |c_n|.$ It thus follows that $\sum_{n=1}^{\infty} m_n = \sum_{n=1}^{\infty} |c_n|$ and therefore converges. Thus, by the Weierstraß M-Test, it follows that $\sum f_n$ is uniform convergent and thus $\sum f_n \stackrel{unif}{\to} u.$

    Thus, by the Term-by-Term Continuity Theorem, $u$ is continuous.
\end{proof}

\noindent
\textbf{We next show that $u$ solves the heat equations.}

\begin{proof}
    We first claim that the boundary conditions are met, i.e.
    \[u(t,0)=u(t,1)=0.\]
    This is as for any $n\in\mathbb{N}$ we have that
    \[\sin(n\pi0)=\sin(n\pi1) =0,\]
    and therefore
    \[u(t,0) = u(t,1) = \sum_{n=1}^{\infty} 0 = 0\]
    as desired.

    Now we show that
    \begin{equation} \label{eq:truth}
        \frac{\partial }{\partial t}u = \frac{\partial^2}{\partial x^2}u.
    \end{equation}
    As each $f_n$ is a solution to the heat equation, it is the case that
    \begin{equation} \label{eq:eq}
        \frac{\partial }{\partial t}f_n = \frac{\partial^2}{\partial x^2}f_n = -c_n(\pi n)^2e^{-(\pi n)^2t}\sin(n\pi x)
    \end{equation}
    for arbitrary $n$.

    Now consider some $c>0,$ we will show that \eqref{eq:truth} is true for $(t,x)\in (c, \infty)\times [0,1].$ With the fixed $c,$ we first define sequence $(z_n)$ with
    \begin{equation*}
        z_n = \frac{(\pi n)^2}{e^{(\pi n)^2c}}.
    \end{equation*}
    It follows by properties of exponentiation that for any $t>c$ and any $n\in\mathbb{N}$ we have
    \begin{equation} \label{eq:tz}
        \left | \frac{(\pi n)^2}{e^{(\pi n)^2t}} \right | \le |z_n|.
    \end{equation}
    Furthermore, as $z_n \to 0$ (the nonrigorous explanation being that exponentiation grows faster than quadratic), there exists $N\in\mathbb{N}$ s.t. $n\ge N$ implies that
    \begin{equation} \label{eq:less}
        |z_n| < 1.
    \end{equation}

    Now going back to $f_n,$ for any arbitrary $n$ we have that
    \begin{alignat*}{2}
        \left | \frac{\partial }{\partial t}f_n \right | &= |-c_n(\pi n)^2e^{-(\pi n)^2t}\sin(n\pi x)|\\
        &\le |c_n(\pi n)^2e^{-(\pi n)^2t}| && \text{as $-1 \le \sin x \le 1$ for all $x$}\\
        &= \left |c_n\frac{(\pi n)^2}{e^{(\pi n)^2t}} \right |\\
        &\le |c_n z_n| && \text{by \eqref{eq:tz}.}
    \end{alignat*}
    If we further only consider $n \ge N,$ we also have that
    \begin{alignat*}{2}
        \left | \frac{\partial }{\partial t}f_n \right | &\le |c_n| \qquad&& \text{by \eqref{eq:less}.}
    \end{alignat*}
    Thus, if we define the sequence $(M_k),$ where
    \begin{equation*}
        M_k = \begin{cases*}
            |c_k z_k| & $k < N$\\
            |c_k| &o.w.
        \end{cases*},
    \end{equation*}
    it follows that $\sum M_k$ converges (as it differs from $\sum |c_k|$ only in the first $N-1$ terms). Similarly, by construction, we have that $|f_k| < M_k$ for any $k.$ Therefore, by the Weierstraß M--test, we have that $\sum \frac{\partial}{\partial t}f_n$ converges uniformly. Thus, by the Term--by--Term Differentiability Theorem, it converges to $\frac{\partial}{\partial t} u.$

    By \eqref{eq:eq}, it follows that $\sum \frac{\partial^2}{\partial x^2} f_n$ also converges to $\frac{\partial}{\partial t} u.$ Thus, to prove \eqref{eq:truth}, it suffices to show that $\sum \frac{\partial^2}{\partial x^2} f_n$ converges to $\frac{\partial^2}{\partial x^2} u.$

    To do so, we need only repeat the same method as above (with Weierstraß M--test and Term--by--Term differentiability---which I omit for brevity) to show that $\sum \frac{\partial}{\partial x} f_n \stackrel{unif}{\to} \frac{\partial}{\partial x} u,$ and then  \emph{again} to show that $\sum \frac{\partial^2}{\partial x^2} f_n \stackrel{unif}{\to} \frac{\partial^2}{\partial x^2} u$ as desired.

    Thus, we have that \eqref{eq:truth} is true for $(t,x)\in (c, \infty)\times [0,1].$ As $c$ was arbitrary, it is also true for $(t,x)\in (0, \infty)\times [0,1].$ Thus $u$ solves the heat equations as desired.
\end{proof}

\newpage
\section*{3}
\begin{proof}
    When $n=m,$ we have that
    \[\sin(n\pi x)\sin(m\pi x)= \sin^2(n\pi x).\]
    Now using the trig. identity:
    \[\sin^2(x)=\frac{1-\cos(2x)}{2},\]
    it follows that
    \begin{alignat*}{2}
        2\int_{0}^{1} \sin(n\pi x)\sin(m\pi x)dx &=2\int_{0}^{1} \sin^2(n\pi x)dx\\
        &= 2\int_{0}^{1} \frac{1-\cos(2n\pi x)}{2} dx\\
        &= 2\left. \left [\frac{1}{2}x - \frac{\sin(2n\pi x)}{4n\pi}\right ]  \right\vert_0^1\\
        &= 1.
    \end{alignat*}

    When $n\neq m,$ we can use the trig. identity
    \[\sin x\sin y = \frac{1}{2}[\cos(x-y)+\cos(x+y)].\]
    Specifically, it follows that
    \begin{alignat*}{2}
        2\int_{0}^{1} \sin(n\pi x)\sin(m\pi x)dx &= 2\int_{0}^{1} \frac{1}{2}[\cos(n\pi x-m\pi x)+\cos(n\pi x+m\pi x)]dx\\
        &= \int_{0}^{1} \cos((n-m)\pi x)+\cos((n+m)\pi x) dx\\
        &= \left. \frac{\sin((n-m)\pi x)}{(n-m)\pi } + \frac{\sin((n+m)\pi x)}{(n+m)\pi } \right\vert_0^1\\
        &= 0.
    \end{alignat*}
\end{proof}

\noindent
\textbf{We next want to show that $c_n = \hat{u}_{0,n}$ for arbitrary $n\in\mathbb{N}$.}

\begin{proof}
    Given arbitrary $n\in\mathbb{N},$ we have that
    \begin{alignat*}{2}
        \hat{u}_{0,n} &= 2 \int_{0}^{1}\sin(n\pi x)u_0(x)dx\\
        &=  2 \int_{0}^{1}\sin(n\pi x) \sum_{m=1}^{\infty}c_m\sin(m\pi x)\\
        &= \sum_{m=1}^{\infty} 2 \int_{0}^{1}\sin(n\pi x)c_m\sin(m\pi x) &&\text{by Term--by--Term Int. Thm\footnotemark}\\
        &= 2 \int_{0}^{1}\sin(n\pi x)\sin(n\pi x) c_n + \sum_{m=1, m\neq n}^{\infty} 2 \int_{0}^{1}\sin(n\pi x)c_m\sin(m\pi x)\\
        &= c_n &&\text{by 1st part of this problem}
    \end{alignat*}
    \footnotetext{Specifically as the partial sums of $u(t,x)$ are continuous, and we assume they converge uniformly.}
\end{proof}

\end{document}