\documentclass[10pt]{article}
\setlength{\textwidth}{6.3in}
\setlength{\textheight}{9in}
\setlength{\oddsidemargin}{0in}
\setlength{\evensidemargin}{0in}
\setlength{\topmargin}{-.5in}
%\parindent=0in
\linespread{1.3}
\usepackage{ mathrsfs }
\usepackage{amsthm}
\usepackage{ amssymb }
\usepackage{graphicx}
\newtheorem{theorem}{Theorem}[section]
\newtheorem{lemma}[]{Lemma}
\newtheorem{definition}[]{Definition}

\usepackage{amsmath}
\usepackage{amsfonts}
\usepackage{fancyhdr}
\usepackage{nicematrix}

\pagestyle{fancy}
\headheight = 14.5pt
\lhead{Exam 2 Practice, Thomas Zeng }
\rhead{CS 254, Winter 2023}
\cfoot{\thepage}

\begin{document}

\noindent
\textbf{A. }Show that a PDA with a queue instead of stack is equivalent to a Turing Machine.

\newpage

\noindent
\textbf{B. }A Multi-dimensional Turing machine is a TM with an infinite tape that is two dimensional. The head of the TM starts at the bottom left corner of the tape. At each step of the computation, the head of the TM can either move left, right, up or down. Give a proof sketch showing that a Multi-dimensional TM is equivalent to a standard (single-tape one-dimensional) TM.



\newpage

\noindent
\textbf{C. }Define the language $\textit{HALTE}_{TM}=\{\langle M\rangle: M \text{ halts on empty string }\epsilon\}$. Show this language is undecidable.\\


\newpage

\noindent
\textbf{D. } Assume (for contradiction) that $\textit{HALT}_{TM}$ is decidable. Show that $\overline{\textit{HALT}}_{TM}$ i.e. the complement of $\textit{HALT}_{TM}$ would then be decidable. Therefore, is it true that $\overline{\textit{HALT}}_{TM}\le_m \textit{HALT}_{TM}?$ Justify your answer. \\

\newpage



\noindent
\textbf{E. } In class, we showed that $\overline{\textit{HALT}}_{TM} \le_m \textit{INFINITE}_{TM}$. Alternatively, we could have used the languages $\textit{EMPTY}_{TM}$ or $\overline{A}_{TM}.$ Thus show that both $\textit{EMPTY}_{TM} \le_m \textit{INFINITE}_{TM}$ and $\overline{A}_{TM}\le_m \textit{INFINITE}_{TM}.$\\




\newpage

\noindent
\textbf{F. }As a remainder, below is a proof for Rice's Theorem (using $\textit{HALT}_{TM}$).

\begin{proof}
    Let $R$ be a $TM$ that rejects everything i.e. $L(R)=\emptyset.$ Let $A$ be a nontrivial property of recognizable language. WLOG, assume  $\langle R\rangle\notin A$ (otherwise just choose $A$ to be the complement).\\

Let $K$ be a TM s.t. $\langle K\rangle\in A.$ Assume for sake of contradiction, $H$ is a decider for $A.$ We can then construct a decider $D$ that decides $HALT_{TM}.$ Specifically, $D$ on input $\langle M, w\rangle$ does the following:
\begin{enumerate}
    \item Build TM $N$ that on input $x$\begin{enumerate}
        \item Save $x$ on its tape
        \item Run $M$ on $w$
        \item Run $K$ on $x,$ outputting what $K$ outputs.
    \end{enumerate}
    \item Run $H$ on $\langle N\rangle,$ and output whatever $H$ outputs 
\end{enumerate}
We note that $L(N) = L(K)$ if and only if $M$ halts on $w,$ i.e. $\langle M, w\rangle\in HALT_{TM}$ and otherwise $L(N) = L(R) = \emptyset.$ Thus, $\langle N\rangle\in A$ if and only if  $\langle M, w\rangle\in HALT_{TM}.$ Resultingly, if $H$ accepts $\langle N\rangle,$ then $\langle M, w\rangle$ must have halted. Similarly, if $H$ rejects $\langle N\rangle,$ then $\langle M, w\rangle$ must not have halted. Thus $D$ decides $HALT_{TM}.$\\
\end{proof}
% emptyy tm
%   try all strings up to length of input, if any string accept, accept, else reject

% A_tm complement
%   try all strings up to length of input, if not accepted, then accept the string

\noindent
Rice's Theorem depends on two assumptions, namely that the property of languages is nontrivial and that it is semantic. How would the above proof break down if the property of language was trivial? Similarly, how would it break down if the property of language was not semantic?


\newpage
\noindent
\textbf{A. }

\begin{proof}
    I am too lazy to write this out. But here is an online solution to this question: 
    
    \texttt{https://cseweb.ucsd.edu/classes/sp07/cse105/hw5soln.pdf}
    % ($\Rightarrow$) 
    % % Showing that a PDA with a queue has an equivalent to a TM is the easy direction. 
    % Consider a two tape TM where the first tape stores the input string and the second tape simulates a queue.

    % ($\Leftarrow$) We can simulate a tape with a queue by enqueuing whatever we pop off. We use some special symbol to demarcate the ends of the input tape and similarly use another symbol to represent the location of the tape head. To move the tape head right is trivial. We simply need to consider the case when we hit the symbol demarcating the end of the input tape, at which point we just push onto the queue some character representing 
\end{proof}

\noindent
\textbf{B. }We know there exists a bijective mapping $f:\mathbb{N}\times\mathbb{N}\to\mathbb{N}.$ Therefore, we can map every cell in the two-dimensional to a unique corresponding cell in the one-dimensional tape. Thus, whenever our multi-dimensional TM tells us to move to cell $(i,j)$, our one-dimensional TM simulates it by moving to cell $f(i,j).$ We claim this is possible as we can construct a two-tape TM where the first tape holds the input string and the second tape holds the index of the cell in the first tape our tape head is at. Thus, we always know the index of the cell our tape head is pointing at.

\noindent
\textbf{C. }We can show that $HALT_{TM}\le_{TM}\textit{HALTE}_{TM}.$
\begin{proof}
Assume $N$ decides $\textit{HALTE}_{TM}.$ We now construct a decider $D$ that decides $HALT_{TM}.$ Specifically, on input $\langle M,w\rangle,$ $D$ does the following:
\begin{enumerate}
    \item Construct TM $K$ that does the following on input $x$:\begin{enumerate}
        \item If $x\neq \epsilon,$ reject
        \item If $x= \epsilon,$ run $M$ on $w.$
        \item accept
    \end{enumerate}
    \item Run $N$ on $K.$
    \item If $N$ accepts, then $D$ accepts
    \item If $N$ rejects, then $D$ rejects
\end{enumerate}
\noindent
Thus using $N$ we constructed a decider $D$ for $HALT_{TM}$ -- a contradiction. 
\end{proof}

\noindent
\textbf{D. }Assume $D$ decides $\textit{HALT}_{TM}.$ We construct TM $M$ that simply runs the input on $D$ and outputs the opoosite of what $D$ outputs. Then $M$ would decide $\overline{\textit{HALT}}_{TM}.$

However, it is not true that $\overline{\textit{HALT}}_{TM}\le_m \textit{HALT}_{TM}.$ We know that $\textit{HALT}_{TM}$ is recognizable. Therefore, if such a mapping reduction existed, then $\overline{\textit{HALT}}_{TM}$ would also be recognizable -- which would make $\textit{HALT}_{TM}$ decidable, a contradiction.

\noindent
\textbf{E. }For $\textit{EMPTY}_{TM}$ consider the mapping $\langle M\rangle$ to $\langle M' \rangle = f(\langle M\rangle)$ where $M'$ on input $s$ simulates $M$ on all strings up to length $|s|$ for $|s|$ steps. If no strings are accepted, then $M'$ accepts $s$, else reject.

For $\overline{A}_{TM}$ consider the mapping $\langle M, w\rangle$ to $\langle M' \rangle = f(\langle M, w\rangle)$ where $M'$ on input $s$ simulates $M$ on $w$ for $|s|$ steps. If $w$ is not accepted, then $M'$ accepts $s$, else reject.

\noindent
\textbf{F. }If $A$ was trivial, then we would not be able to find $\langle K\rangle\in A.$ If $A$ was not semantic, then $L(N)=L(K)$ does not necessarily imply that $\langle N\rangle\in A$. 


\end{document}