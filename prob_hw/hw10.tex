\documentclass[10pt]{article}
\setlength{\textwidth}{6.3in}
\setlength{\textheight}{9in}
\setlength{\oddsidemargin}{0in}
\setlength{\evensidemargin}{0in}
\setlength{\topmargin}{-.5in}
%\parindent=0in
\linespread{1.3}
\usepackage{ mathrsfs }
\usepackage{amsthm}
\usepackage{ amssymb }
\usepackage{graphicx}
\newtheorem{theorem}{Theorem}[section]
\newtheorem{lemma}[theorem]{Lemma}

\usepackage{amsmath}
\usepackage{amsfonts}
\usepackage{fancyhdr}
\usepackage{nicematrix}

\pagestyle{fancy}
\headheight = 14.5pt
\lhead{Probability HW10, Thomas Zeng }
\rhead{Math 240, Fall 2022}
\cfoot{\thepage}

\begin{document}
\section*{1}
\subsection*{a}
As each $X_i$ is i.i.d. given $f$ is the pdf for $N(\mu=11,\sigma^2=10^2)$,
\[f_\text{joint}(x_1,...,x_n)=f(x_1)f(x_2)...f(x_n)\]

\subsection*{b}

\begin{align*}
    F_Y(y) &= P(Y\le y)\\
    &= 1 - P(X_1 > y,...,X_n > y)\\
    &= 1 - P(X_1>y)P(X_2>y)...P(X_n>y)\\
    &= 1 - (1-P(X\le y))^n\\
    &= 1 - (1-F_X(y))^n
\end{align*}

\begin{align*}
    f_Y(y) &= F_Y(y)dy\\
    &= 1 - (1-F_X(y))^ndy\\
    &= -n(1-F_X(y))^{n-1}(1-F_X(y)dy)\\
    &= n(1-F_X(y))^{n-1}f_x(y)
\end{align*}

\subsection*{c}
For $n=3,$ we want $F_Y(0)=1-(1-F_X(0))^3\approx 0.354$ using \texttt{1-(1-pnorm(0,11,10))\textasciicircum3}.

\noindent
For $n=30,$ we want $F_Y(0)=1-(1-F_X(0))^{30}\approx 0.987$ using \texttt{1-(1-pnorm(0,11,10))\textasciicircum30}.

The probability increase as we are looking for the minimum. With more measurements, it is less likely that all measurements will be above $0$.

\section*{2}
Let $W = X+Y.$ As $X,Y$ are exponential distributions, the support set of $W$ is all $w>0$.
\begin{align*}
    f_{X+Y}(t)&=\int_0^t f_X(t-y)f_Y(y)dy\\
    &=\int_0^t e^{-(t-y)}\frac{1}{2}e^{-\frac{y}{2}}dy\\
    &= \int_0^t\frac{1}{2}e^{\frac{y}{2}-t}dy\\
    &= e^{\frac{y}{2}-t} \Bigr |_0^t\\
    &= e^{-\frac{t}{2}}-e^{-t}
\end{align*}

\section*{3}
%8.34
\subsection*{a}
As each $X_i$ is exponential, their support is all larger than $0$. Thus the support of $Z$ is all numbers $z > -\log n.$

\begin{align*}
    F_Z(z) &= P(Z\le z)\\
    &= P( \max(X_1,...X_n)\le z+\log n)\\
    &= P(X_1 < z+\log n, ...,X_n<z+\log n)\\
    &= P(X<z+\log n)^n\\
    &= F_X(z+\log n)^n\\
    &= (1- e^{-(z+\log n)})^n\\
    &= \left (1- \frac{e^{-z}}{n}\right )^n,\; z>-\log n
\end{align*}

\subsection*{b}

% WTS $F_Z(z)\to e^{-e^{-z}}$ as $n\to\infty$

\begin{align*}
    \lim_{n\to\infty}\left(1- \frac{e^{-z}}{n}\right )^n &=  \lim_{n\to\infty}\left(1+ \frac{-e^{-z}}{n}\right )^n\\
    &= e^{-e^{-z}}
\end{align*}

\subsection*{c}
Let $N_{120}$ be the number of times it rains heavily in $10$ years ($120$ months). Then $N_{120}\sim\text{Pois}(120).$ We also know that for some $n$ and $t$, $P(\max(X_1,...X_n)\le t)= P(Z\le t -\log n).$ Let $Y$ be the random variable denoting max time between heavy rain in $10$ years. Then:
\begin{align*}
    P(Y >3) &= 1 - P(Y \le3)\\
    &= 1 - \sum_{n=0}^\infty P(N_{120}=n)P(\max(X_1,...X_n)\le 3)\\
    &= 1 - \sum_{n=0}^\infty P(N_{120}=n)P(Z\le 3 -\log n)\\
    &= 1 - \sum_{n=0}^\infty P(N_{120}=n)F_Z(3-\log n)\\
    &= 1 - \sum_{n=0}^\infty P(N_{120}=n)\left (1- \frac{e^{-(3-\log n)}}{n}\right )^n\\
    &= 1 - \sum_{n=0}^\infty \frac{120^ne^{-120}}{n!}\left (1- \frac{e^{-(3-\log n)}}{n}\right )^n\\
    &\approx 0.997\;\text{according to Wolfram Alpha}
\end{align*}

\section*{4}
\subsection*{a}
\begin{align*}
    f_X(x) &= \int_0^1 x+ydy\\
    &= xy + \frac{y^2}{2}\Bigr |_0^1\\
    &= x + \frac{1}{2}
\end{align*}

Thus:
\begin{equation*}
    f_X(x)=\begin{cases}
        x+\frac{1}{2},\;0<x<1\\
        0,\; o.w.
    \end{cases}
\end{equation*}

And as the joint distribution is symmetric with regards to $x,y,$ we thus also have:
\begin{equation*}
    f_Y(y)=\begin{cases}
        y+\frac{1}{2},\;0<y<1\\
        0,\; o.w.
    \end{cases}
\end{equation*}

\subsection*{b}
We know that:
 \[f(y|x)=\frac{f(x,y)}{f_X(x)}.\]

We know this conditional probability is only valid (i.e. nonzero) if $0<x<1$ and $0<y<1.$ Thus we write it as
\begin{equation*}
    f(y|x)=\begin{cases}
        (x+y)/(x+\frac{1}{2}),\;0<x<1,0<y<1\\
        0,\; o.w.
    \end{cases}
\end{equation*}

\subsection*{c}
First we have that $f(y|.4)=(.4+y)/(.4+\frac{1}{2}).$

\begin{align*}
    P(Y\le0.3|X=.4) &= \int_0^{0.3}(.4+y)/(.4+\frac{1}{2})dy\\
    &= \int_0^{0.3} \frac{10}{9}(\frac{2}{5}+y)dy\\
    &= \int_0^{0.3} \frac{4}{9} + \frac{10}{9}ydy\\
    &= \frac{4}{9}y +\frac{5}{9}y^2 \Bigr |_0^{0.3}\\
    &\approx .183 
\end{align*}

\subsection*{d}
\begin{align*}
    P(Y\le0.3)&=\int_0^{0.3} y+\frac{1}{2}dy\\
    &= y^2 + \frac{1}{2}y \Bigr |0^{0.3}\\
    &=0.24
\end{align*}
No, as $P(Y\le 0.3) \neq P(Y\le0.3|X=.4).$

\section*{5}
\begin{equation*}
    P(X=x|Y=2)=\begin{cases}
        0/.125,\; x=0\\
        .05/.125,\; x=1\\
        .05/.125,\; x=2\\
        .025/.125,\; x=3\\
        0,\; \text{o.w.}\\
    \end{cases}
\end{equation*}

To rewrite this, we have
\begin{equation*}
    P(X=x|Y=2)=\begin{cases}
        0,\; x=0\\
        .4,\; x=1\\
        .4,\; x=2\\
        .2,\; x=3\\
        0,\; \text{o.w.}\\
    \end{cases}
\end{equation*}

\section*{6}
\subsection*{a}

On its support set, we have
\begin{align*}
    f_X(x) &= \int_x^1 2dy\\
    &= 2y \Bigr |_x^1\\
    &= 2 - 2x
\end{align*}

Thus we have
\begin{equation*}
    f_X(x) = \begin{cases}
        2-2x,\;0<x<1\\
        0,\;\text{o.w.}
    \end{cases}
\end{equation*}
\subsection*{b}
As in problem $4,$ we first note that $f(y|x) = f(x,y)/f_X(x).$
The support is $0<x<y<1$ and $0$ otherwise. More formally
\begin{equation*}
    f(y|x) = \begin{cases}
        2/(2-2x),\;0<x<y<1\\
        0,\;\text{o.w.}
    \end{cases}
\end{equation*}
This looks like a beta distribution with $a=1,b=2.$

\section*{7}
\subsection*{a}
We know that $f(x,y) = f(x|y)f(y).$ As exponential distributions have support set of values greater than or equal to $0.$ The support here is $x\ge0,y\ge0.$ Therefore:
\begin{equation*}
    f(x,y)=\begin{cases}
        ye^{-yx}16e^{-16y},\;x\ge0,y\ge0\\
        0,\;\text{o.w.} 
    \end{cases}
\end{equation*}

\subsection*{b}

For $x>0$ we have that:
\begin{align*}
    f_X(x) &= \int_0^\infty f(x,y)dy\\
    &=\int_0^\infty  ye^{-yx}16e^{-16y}dy\\
    &=\int_0^\infty 16ye^{-(16+x)y}dy\\
    &= 16\int_0^\infty ye^{-(16+x)y}dy \quad \text{(proportional to Gamma with $a=2,\lambda=16+x$)}\\%a=2
    &= 16\int_0^\infty\frac{(16+x)^2}{(16+x)^2}ye^{-(16+x)y}dy\\
    &= \frac{16}{(16+x)^2}\int_0^\infty (16+x)^2ye^{-(16+x)y}dy\\
    &=\frac{16}{(16+x)^2}
\end{align*}

\subsection*{c}
As $Y$ follows exponential distribution, we have $y\ge0.$
\begin{align*}
    f(y|x) &= f(x,y)/f_X(x)\\
    &= 16ye^{-(16+x)y}/\left(\frac{16}{(16+x)^2}\right)\\
    &= 16ye^{-(16+x)y}\frac{(16+x)^2}{16}\\
    &=(16+x)^2ye^{-(16+x)y}
\end{align*}
This is precisely the distribution Gam$(2,16+x).$
\end{document}