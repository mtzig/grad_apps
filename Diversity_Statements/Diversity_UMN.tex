\documentclass[12pt]{article}
\setlength{\textwidth}{6.3in}
\setlength{\textheight}{9in}
\setlength{\oddsidemargin}{0in}
\setlength{\evensidemargin}{0in}
\setlength{\topmargin}{-.5in}
%\parindent=0in
\linespread{1.3}
\usepackage{ mathrsfs }
\usepackage{amsthm}
\usepackage{ amssymb }
\newtheorem{theorem}{Theorem}[section]
\newtheorem{lemma}[theorem]{Lemma}

\usepackage{amsmath}
\usepackage{amsfonts}
\usepackage{fancyhdr}
\pagestyle{fancy}
\headheight = 14.5pt
\lhead{Diversity Statement - Thomas Zeng}
\rhead{University of Minnesota}
\cfoot{\thepage}

\usepackage{tikz}
\usetikzlibrary{positioning}
\begin{document}

%4000 character limit (without space)

% The statement should cover how your background, experiences, and achievements will contribute to the University's and our Department’s goal of promoting excellence through diversity and inclusion. If applicable, you should mention hardships or obstacles that you have overcome to complete your undergraduate education. Some University of Minnesota graduate student applicants will be considered for funding opportunities, like the DOVE Fellowship, based upon their diversity statement.

% Potential areas to address include:

% hardships you have faced and overcome
% family background
% first-generation college student
% non-traditional attributes for computer science/data science
% your diverse identities (examples: ethnic, racial, economic, or educational backgrounds and experiences, gender identity, sexual orientation)
% your commitment to working toward achieving equity and enhancing diversity within the field of computing, STEM, or within your community
% geographic diversity
% community involvement
% multilingual skills
% international travel which changed your perspective
% special talents
% anything unique you would contribute to the program


As an minority in the United States -- specifically a Chinese American, my experiences growing up has emphasized to me the need of promoting diversity in education. More specifically, I am a second-generation immigrant who grew up in a predominantly-white neighborhood and have first hand experience of the difficulties one can experience when they are ``different" from the majority group -- both through the inadvertent micro-aggression and cultural differences. At the same time, the minority experience has afforded me the ability to have a greater appreciation for multiculturalism and empathy for other minority groups. I've lived in China for a year when I was younger, and this has given me a different perspective from some of my pears who grew up entirely in the United States.

At the same time, I am aware that although I racially am a minority, in many other ways I have been greatly privileged. Firstly, as a man in the STEM field, I do not have to worry as much about work discrimination that other women may face. Similarly, I am financially secure enough with my parent's support, that I do not have to worry monetarily about school and thus can explore and find my passion at my own pace. My sexual orientation and gender identity also all fall within the societally perceived ``norm". This has made me cognizant of the fact that there are many societal issues that others deal with on a daily basis that I do not have to worry about. Thus, my identity both as a minority in some ways and societally privileged in other ways, has distilled in me the importance of ensuring accessibility and equity.

To promote inclusion, I am currently trying to make machine learning more accessible to others. Specifically, I hold machine learning workshops at the data science club at my college open to everyone. Through this I hope to lower the barrier of entry and bring more people into this field. Similarly, this has partially motivated my research interest of fair and explainable ML models. I believe ML will have an outsized impact on society in the future, and it is important to guarantee that the models will not discriminate against minority groups.

I have also attempted to become a more open-minded person by interacting with more international students at my school. And hence a majority of my friends are from all over the world. Through them, I have gained exposure to various different cultures and have gained a better picture of the vastly different experiences that people can have. I believe this sensitivity is also something of value that I can bring to the University of Minnesota.
% I am asian american -- a minority in America
    % Priviledged -- my parents have been willing to support me -- allow me to find and develop my own passion
    % I believe equity and fairness is important -- part of the reason why this is the line of research I want to pursue in machine learning -- insure models are equitable to people of all types
    % Second generation immigrant
        % most of my friends are international students -- broaden my understanding of people from different cultures
    % making machine learning and deep learning more acessible to others, I hold workshops at the datascience club in my school to bring more people into the field.
\end{document}
