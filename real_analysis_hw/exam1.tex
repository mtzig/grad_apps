\documentclass[10pt]{article}
\setlength{\textwidth}{6.3in}
\setlength{\textheight}{9in}
\setlength{\oddsidemargin}{0in}
\setlength{\evensidemargin}{0in}
\setlength{\topmargin}{-.5in}
%\parindent=0in
\linespread{1.3}
\usepackage{ mathrsfs }
\usepackage{amsthm}
\usepackage{ amssymb }
\usepackage{graphicx}
\newtheorem{theorem}{Theorem}[]
\newtheorem{lemma}[]{Lemma}
\newtheorem{definition}[]{Definition}

\usepackage{amsmath}
\usepackage{amsfonts}
\usepackage{fancyhdr}
\usepackage{nicematrix}
\usepackage{soul}

\newcommand*{\tabulardef}[3]{\begin{tabular}[t]{@{}lp{\dimexpr\linewidth-#1}@{}}
    #2&#3
  \end{tabular}}
\pagestyle{fancy}
\headheight = 14.5pt
\lhead{Real Analysis Exam 1 (Take home) }
\rhead{Math 321, Winter 2023}
\cfoot{\thepage}

\newcounter{relctr} %% <- counter for relations
\everydisplay\expandafter{\the\everydisplay\setcounter{relctr}{0}} %% <- reset every eq
\renewcommand*\therelctr{\alph{relctr}} %% <- label format

\newcommand\labelrel[2]{%
  \begingroup
    \refstepcounter{relctr}%
    \stackrel{\textnormal{(\alph{relctr})}}{\mathstrut{#1}}%
    \originallabel{#2}%
  \endgroup
}
\AtBeginDocument{\let\originallabel\label} %% <- store original definition


\begin{document}

\section*{Problem 1}
\subsection*{(a)}
This sequence diverges. First, we assert that
\begin{equation}
    \sum_{n=1}^\infty \frac{1}{2n} \label{eq:series}
\end{equation}
 must diverge. Specifically we would argue that if the series in \eqref{eq:series} converged, then by the Algebraic Limit Theorem for Series, the harmonic series
 \begin{equation*}
    \sum_{n=1}^\infty \frac{1}{n} = \sum_{n=1}^\infty2\left (\frac{1}{2n}\right )
 \end{equation*}
 would also converge. This is a contradiction as we know (as proved in the textbook) that the Harmonic series is divergent. Thus the series in \eqref{eq:series} is divergent.

 Next, by algebra we have that
 \begin{equation*}
    0<\frac{1}{2n}<\frac{n+1}{2n}
 \end{equation*}
 for all $n\in\mathbb{N}.$ Therefore by the Comparison Test, we have that $\sum_{n=1}^\infty\frac{n+1}{2n}$ must diverge.

\subsection*{(b)}
This sequence converges.
We know that the p-series 
\begin{equation*}
    \sum_{n=1}^\infty 1/n^3
\end{equation*}
converges as the denominator is raised to a power greater than $1$.
Next by algebra, we know that
\begin{equation*}
    0<\frac{1}{n^3+1}<\frac{1}{n^3}
 \end{equation*}
 for all $n\in\mathbb{N}.$ Thus, by the comparison test, 
 \begin{equation*}
    \sum_{n=1}^\infty \frac{1}{n^3+1}
 \end{equation*}
also converges. 
Now by more algebraic manipulation (namely dividing the numerator and denominator by $n^{2/3}$), we assert that
\begin{equation} \label{eq:equiv}
    \frac{n^{2/3}}{n^2+n^{2/3}}=\frac{1}{n^3+1}.
\end{equation}
Therefore, by the equality in \eqref{eq:equiv}, the series
\begin{equation*}
    \sum_{n=1}^\infty \frac{n^{2/3}}{n^2+n^{2/3}} = \sum_{n=1}^\infty\frac{1}{n^3+1}
\end{equation*}
converges. Lastly as
\begin{equation*}
    \sum_{n=1}^\infty \frac{2n^{2/3}}{n^2+n^{2/3}} = \sum_{n=1}^\infty 2\left(\frac{n^{2/3}}{n^2+n^{2/3}}\right).
\end{equation*}
Thus, by the Algebraic Limit Theorem for series, $\sum_{n=1}^\infty \frac{2n^{2/3}}{n^2+n^{2/3}}$ must also converge.

\newpage
\section*{Problem 2}
\begin{proof}
    By definition of bounded, there exists some $M$ s.t. 
    \begin{equation} \label{eq:bound}
        |y_n|\le M
    \end{equation}
    for all $n\in\mathbb{N}.$ We now assert that $(x_ny_n)$ converges to $0$. Fix $\epsilon>0.$ Then as $(x_n)\to 0,$ there exists $N\in\mathbb{N}$ s.t. for all $n\ge N$ we have
    \begin{equation}
        |x_n|<\epsilon/M. \label{eq:epsilon}
    \end{equation}
    Now fix some $k\ge N.$ We thus have
    \begin{align}
        |x_k||y_k| &< \epsilon/M(M) \label{eq:mult}\\
        |x_ky_k| &<\epsilon \label{eq:homo}\\
        |x_ky_k-0|&<\epsilon \nonumber
    \end{align}
    where \eqref{eq:mult} follows from \eqref{eq:bound} and \eqref{eq:epsilon}; and \eqref{eq:homo} follows as absolute value is a homorphism under multiplication.
    Thus, by definition of convergence, $(x_ny_n)$ also converges to $0$. That is $(x_ny_n)$ converges.
\end{proof}

\newpage
\section*{Problem 3}
The intuition behind this proof will be that we can put a bound on where the supremum would be (if it exists). We can methodically shrink these bounds by performing a sort of ``binary search'' which will yield the supremum.

We first describe a method of generating a sequence of nested intervals for any nonempty set bounded above. The idea will be that each interval is guaranteed to contain the supremum (if it exists) and the length of these nested intervals converge to $0$.

\begin{definition}[Sup-Intervals] \label{def:sup}
    Let $A\subseteq \mathbb{R}$ be some nonempty set bounded above by $b$.  We define a \textbf{Sup-Intervals} of $A$ which we denote $(I_n)$ as a sequence of nested intervals
    \[I_1\supset I_2\supset I_3 \supset \cdots \]
    meeting the following conditions.
    Firstly, $I_1 = [\ell_1, b_1]$ where $\ell_1 = a-1$ for some $a\in A$ and $b_1 = b.$  For $n>1,$ we construct $I_n = [\ell_n, b_n]$ as follows:

    \begin{center}
        \fbox{%
            \begin{minipage}{5in}
            \textbf{To generate} $I_n = [\ell_n, b_n]$: 
            \begin{itemize}
                \item[] \textsc{Let} $m$ be the midpoint of the interval $I_{n-1}$
                \item[] \textsc{If} $m$ is an upper bound of $A:$
                \item[] \quad \textsc{Set} $\ell_n:=\ell_{n-1}$ and $b_n:=m$
                \item[] \textsc{Else}:
                \item[] \quad \textsc{Set} $\ell_n:=m$ and $b_n:=b_{n-1}$
            \end{itemize}
          \end{minipage}
          }
    \end{center}
\end{definition}


Using the definition of Sup-Intervals, we now lay out a few properties that will be relevent to proving the Axiom of Completeness. For completeness (no pun intended), we include proofs for each property (although they should all intuitively follow by our definition of Sup-Intervals). Also, for notational consistency, we will hold constant for the rest of this problem the variables $I_n,\ell_n,b_n$ where for all $n\in\mathbb{N}$ we have $I_n=[\ell_n,b_n].$ 

\begin{lemma} \label{lem:len}
    Let $(I_n)$ be a Sup-Intervals of a set $A$ where $|I_1|=M,$ then we have that $|I_n|=M(1/2)^{n-1}$ for $n\in\mathbb{N}.$
\end{lemma}

\begin{proof}
    This is true by induction. For the base case, $|I_1|=M(1/2)^{1-1}=M.$ For the inductive hypthesis, assume $|I_n|=M(1/2)^{n-1}$ for some $n>1$. By the definition of Sup-Intervals, $I_{n+1}$ is constructed by  bi-secting $I_n$ and choosing one of the resulting intervals. Therefore, using the inductive hypothesis we find that $I_{n+1}=1/2|I_n|=M(1/2)^n.$
\end{proof}


\begin{lemma} \label{lem:b}
    Let $(I_n)$ be a Sup-Intervals of a set $A$, then $b_n$ is an upper bound for $A$ for any $n\in\mathbb{N}.$
\end{lemma}

\begin{proof}
    This is again true by induction. For the base case, $b_1$ is an upper bound by defintion. Now for the inductive hypothesis, assume $b_n$ be an upper bound. If the midpoint $m$ of $I_n$ is an upper bound, then $b_{n+1}=m$ is an upper bound. Otherwise, $b_{n+1}=b_n$ is an upper bound by the inductive hypothesis.
\end{proof}

\begin{lemma} \label{lem:l}
    Let $(I_n)$ be a Sup-Intervals of a set $A$, then for all $n\in\mathbb{N}$ we have that $\ell_n < a$ for some $a\in A$
\end{lemma}

\begin{proof}
    This is also true by induction. For the base case, $\ell_1 = a - 1 < a$ for some $a\in A$ by definition of Sup-Intervals. For the inductive hypothesis, assume for some $n\in\mathbb{N}$ we have that $\ell_n < a$ for some $a\in A.$ By construction of Sup-intervals, if the midpoint $m$ of $I_n$ is not an upper bound then $I_{n+1} = m.$ By negation of the definition of upper bound, there must be some $a\in A$ s.t. $\ell_{n+1}<a.$ If $m$ is an upper bound, then $I_{n+1}=I_{n}.$ Thus using our inductive hypothesis as $I_n<a$ for some $a\in A$ it thus follows that $I_{n+1}<a.$
\end{proof}

\begin{lemma} \label{lem:in}
    Let $(I_n)$ be a Sup-Intervals of a set $A.$ For all $N\in\mathbb{N}$ we have that for any $n\ge N$ that $b_n\in I_N.$
\end{lemma}

\begin{proof}
    This follows from the fact that a Sup-Intervals is by construction a sequence of nested intervals. For any $n\ge N$ we have that $b_n\in I_n$ (specifically as the right endpoint) and therefore $b_n\in I_n \subseteq I_N.$
\end{proof}

\begin{lemma} \label{lem:d}
    Let $(I_n)$ be a Sup-Intervals of a set $A,$ we then have that $b_{n+1}\le b_n$ for all $n\in\mathbb{N}.$
\end{lemma}

\begin{proof}
    Given $b_n,$ we either have $b_{n+1}=b_n$ if the midpoint $m$ of $I_n$ is not an upper bound. Otherwise $b_{n+1}=m < b_n.$
\end{proof}


\begin{theorem}[\st{Axiom} Theorem of Completeness]
    Let $A\subseteq \mathbb{R}$ be some nonempty set bounded above by some $b\in\mathbb{R}.$ Then $A$ has a supremum.
\end{theorem}

\begin{proof}
   Let $(I_n)$ be a Sup-intervals of $A.$
    We now use the $b_n$'s of each $I_n$ to  construct the sequence $(b_n) = (b_1, b_2,\cdots).$ We claim that this sequence converges to the supremum of $A$. To do so, we first show this sequence is Cauchy and therefore convergent.

    Denote $|I_0|=M.$
    % By our construction, the length of $I_n$ is $M(1/2)^{n-1}.$ Similarly, for any $k\ge n$ it must be that $b_k\in I_n.$
    Now fix $\epsilon>0.$ As we know that $(1/2^n)\to 0,$ there exists some $N$ s.t. for all $n\ge N$ we have
    \begin{align}
        |1/2^n - 0| &< \epsilon/M \label{eq:convdef}\\
        |1/2^n| &< \epsilon/M \nonumber\\
        1/2^n &< \epsilon/M \label{eq:interval}\\
        M(1/2^n) &< \epsilon \label{eq:Mep}
    \end{align}
    where \eqref{eq:convdef} is true by definition of convergence, and \eqref{eq:interval} is true since $1/2^n$ is always greater than $0$ for any $n\in\mathbb{N}.$
    It thus directly follows that for any $n\ge N$ we have
    \begin{align}
        |I_n| &= M(1/2^n) \label{eq:lemma}\\
        &< \epsilon \label{eq:Mepr}
    \end{align}
    where \eqref{eq:lemma} is true by Lemma \ref{lem:len} and \eqref{eq:Mepr} is true by \eqref{eq:Mep}.
    Now for any $n,m\ge N,$ by Lemma \ref{lem:in}, we have that $b_n,b_m\in I_n.$
    It thus follows that
    \begin{align}
        b_n,b_m\in I_N &\Rightarrow
        |b_n-b_m| \le |I_N| \nonumber\\
        &\Rightarrow|b_n-b_m| < \epsilon \label{eq:cauch}
    \end{align} 
    where \eqref{eq:cauch} is true by \eqref{eq:Mepr}.
    Thus, $(b_n)$ is Cauchy by definition. Therefore, it is convergent by the Cauchy Criterion. Hence $(b_n)\to s$ for some  $s\in\mathbb{R}.$

    We now show that $s = \sup A.$ To do this, we first show $s$ is an upper bound of $A$ by contradiction. Assume $s$ is not an upper bound. By negation of definition, there is some $a\in A$ s.t. 
    \begin{equation} \label{eq:f3c}
        a > s.        
    \end{equation}
    Now let $\epsilon = a - s.$ As $(b_n)$ is convergent, there must be some $b_k \in V_\epsilon(s).$ Furthermore by Lemma \ref{lem:d}, we know that $(b_n)$ is decreasing. Thus it must be that 
    \begin{equation} \label{eq:f1c}
        s<b_k,
    \end{equation}
    as otherwise we would have the infintite subseqeunce $(b_{k+1}, b_{k+2},\cdots)$ outside of $V_{|s-b_k|}(s)$ which contradicts the Topological definition of convergence. Therefore, we have that
    \begin{align}
        b_k - s &= |b_k-s| \label{eq:f1}\\
        b_k - s  &< \epsilon \label{eq:f2}\\
        b_k-s&< a - s \label{eq:defep}\\
        b_k &< a \label{eq:f3}
    \end{align}
    where \eqref{eq:f1} is true by \eqref{eq:f1c}; \eqref{eq:f2} is true by our selection of $b_k$; and \eqref{eq:defep} is true by our definition of $\epsilon.$
    Hence, we have that \[s<b_k<a\] by \eqref{eq:f3c}, \eqref{eq:f1c} and \eqref{eq:f3}. However, by Lemma \ref{lem:b}, $b_k$ is an upper bound. Thus, it is a contradiction that $b_k<a$. Therefore, $s$ must be an upper bound.

    Next, we show that for any $\epsilon > 0$ we have that there is an $a\in A$ s.t. $s-\epsilon<a.$
    % for any upper bound $u$ of $A,$ it holds that $s\le u.$
    % We do this again by contradiction. Assume $u$ is an upper bound where $u<s.$ Then let $\epsilon=(s-u)/2.$
    We first fix $\epsilon>0.$ 
    Then by same method as in \eqref{eq:Mepr} we can find some $N_1$ s.t. for all $n\ge N_1$ we have
    \begin{equation} \label{eq:e1}
        |I_{n}|<\epsilon/2.
    \end{equation}
    Similarly, as $(b_n)$ converges to $s,$ we can find some $N_2$ s.t. for all $n\ge N_2$ we have
    \begin{equation} \label{eq:e2}
        |b_{n}-s|<\epsilon/2.
    \end{equation}
     Let $N = \max\{N_1, N_2\}.$ By the triangle inequality, it must be that
    \begin{align}
        |\ell_N -s|&\le |\ell_N-b_N| + |b_N-s| \nonumber\\
        &<\epsilon/2 + \epsilon/2 \label{eq:ee}\\
        &< \epsilon \label{eq:finalepsilon}.
        \end{align}
    where \eqref{eq:ee} is true by \eqref{eq:e1} and \eqref{eq:e2}.  Now as $s$ is an upper bound, it must be that 
    \begin{equation} \label{eq:lno}
       \ell_N < s
    \end{equation}
    since by Lemma \ref{lem:l} we know that $\ell_N$ is not an upper bound.
    Therefore, by  \eqref{eq:lno} it follows that
    \[|\ell_N-s|=s-\ell_N.\]
    Thus we can rewrite the inequality in \eqref{eq:finalepsilon} as
    \begin{equation} \label{eq:ffep}
        s-\ell_N < \epsilon.
    \end{equation}
    We thus have that
    \begin{equation} \label{eq:last}
        s-\epsilon \labelrel<{e1} \ell_N\labelrel<{e2} a 
    \end{equation}
    for some $a\in A,$ where \eqref{e1} is true by \eqref{eq:ffep}, and \eqref{e2} is true by Lemma \ref{lem:l}. Thus by Lemma 1.3.8, $s$ is the supremum of $A$.
\end{proof}
        
\end{document}