\documentclass[10pt]{article}
\setlength{\textwidth}{6.3in}
\setlength{\textheight}{9in}
\setlength{\oddsidemargin}{0in}
\setlength{\evensidemargin}{0in}
\setlength{\topmargin}{-.5in}
%\parindent=0in
\linespread{1.3}
\usepackage{ mathrsfs }
\usepackage{amsthm}
\usepackage{ amssymb }
\usepackage{graphicx}
\newtheorem{theorem}{Theorem}[section]
\newtheorem{lemma}[theorem]{Lemma}

\usepackage{amsmath}
\usepackage{amsfonts}
\usepackage{fancyhdr}
\usepackage{nicematrix}

\pagestyle{fancy}
\headheight = 14.5pt
\lhead{Probability HW8, Thomas Zeng }
\rhead{Math 240, Fall 2022}
\cfoot{\thepage}

\begin{document}
\section*{1}
\subsection*{a}
Let's define $X$ as the random variable of the noise term.

Given Alice sends a $0$, Bob will interpret correctly when $X<0.5$. Given Alice sends a $1$, Bob will interpret correctly when $X\ge-0.5.$ As a normal distribution is symmetric, $P(X<0.5)=P(X>-0.5)=P(X\ge-0.5)$ and Bob's probability of correct interpretation is independent of what Alice sends. Thus Bob will interpret correctly with probability $P(X<0.5)\approx0.691$ using \texttt{pnorm(0.5)}.

\subsection*{b}
If $\sigma$ is small Bob's probability of correct interpretation increases. Inversely, if $\sigma$ is large Bob's probability of correct interpretation decreases. This makes sense intuitively as smaller $\sigma$ means less spread of the distribution and thus less noise.

\section*{2}
\subsection*{a}
$Y=cX$ is a linear transformation and thus invertible. Hence $P(Y<k)=P(X<k/c)$. Therefore $f_Y(k)=F_Y(k)dy=F_X(k/c)dy=f_X(k/c)/c.$ As $X$ is an exponential distribution, $f_Y(k)=f_X(k/c)/c=\lambda e^{-\lambda k/c}/c.$ This is the pdf of an exponential distribution with rate $\lambda /c.$

\subsection*{b}
$W=cX+d$ is a linear transformation and thus invertible. Thus we can compute $f_W$ in the same way.
\begin{align*}
    P(W<k)&=(X<\frac{k-d}{c})\\
    f_W(k)&=f_x\left (\frac{k-d}{c}\right )/c\\
    &=\lambda e^{-\lambda (k-d)/c}/c
\end{align*}
As $f_W$ is not the pmf for an exponential distribution, $W$ is thus not an exponential distribution.

\section*{3}
%7.24
\subsection*{a}
This is a Poisson distribution $X\sim\text{Pois}(21.7).$ We want $P(X>25)=1-P(X\le25)\approx0.20$ using \texttt{1-ppois(25,21.7)}.
\subsection*{b}
We have $S_{10}\sim\text{Gamma}(10, 21.7/31).$ We want $P(S_{10}<15)\approx0.60$ using \texttt{pgamma(15,10,21.7/31)}
\subsection*{c}
This is $N_7\sim\text{Pois}(21.7/31\times7).$

$E[N_7]=\lambda=27.7/31\times7=4.9$
$\sigma = \sqrt[]{V[N_7]}=\sqrt[]{E[N_7]}\approx2.21$

\subsection*{d}
This is $S_1=\text{Exp}(21.7/31).$
$E[S_1]=\frac{1}{\lambda}=31/21.7\approx1.43.$
$\sigma=|\frac{1}{\lambda}|\approx1.43.$

\section*{4}
$P(Y<k)=P(-k<X<k)$ and thus $F_Y(k)=F_X(k)-F_X(-k)$.
\begin{align*}
    f_Y(k) &= F_Y(k)\frac{d}{dk}\\
    &= F_X(k)-F_X(-k)\frac{d}{dk}\\
    &= f_X(k)+f_X(-k)\\
    &=\frac{1}{\sigma\sqrt[]{2\pi}}e^{-k^2/(2\sigma^2)}+\frac{1}{\sigma\sqrt[]{2\pi}}e^{-(-k)^2/(2\sigma^2)}\\
    &= 2\frac{1}{\sigma\sqrt[]{2\pi}}e^{-k^2/(2\sigma^2)}\\
    &= 2f_X(k)
\end{align*}

\section*{5}
%4.18
\subsection*{a}
As we are sampling with replacement, each number has probability $0.25$ of being selected. And thus any two numbers (in that order) has probability $0.25*0.25 = \frac{1}{16}$ of being selected.

Thus, the joint pmf would be the following:

\begin{center}
    \begin{NiceTabular}{ccccc}[vlines]
        \Hline
        \diagbox{X}{Y} & 1   &  2   &  3 & 4\\
        \Hline
        1 & $\frac{1}{16}$& $\frac{1}{16}$& $\frac{1}{16}$& $\frac{1}{16}$\\
        \Hline
        2 & $\frac{1}{16}$& $\frac{1}{16}$& $\frac{1}{16}$& $\frac{1}{16}$\\
        \Hline
        3 & $\frac{1}{16}$& $\frac{1}{16}$& $\frac{1}{16}$& $\frac{1}{16}$\\
        \Hline
        4 & $\frac{1}{16}$& $\frac{1}{16}$& $\frac{1}{16}$& $\frac{1}{16}$\\
        \Hline
    \end{NiceTabular}
\end{center}

\subsection*{b}
$P(X\le Y)$ would be summing the probability in the upper triangle of the pmf table above i.e. $10 \times \frac{1}{16}=\frac{5}{8}.$

\subsection*{c}
There are two cases. If $x\in X$ and $y\in Y$ are different, then we select $x\in X$ with probability $\frac{1}{4}$ and $y\in Y$ with probability $\frac{1}{3}.$ Thus $P(xy)=\frac{1}{12}.$ If $x=y$ then as we are sampling without replacement, $P(y)=0$ and thus $P(xy)=0$.

\begin{center}
    \begin{NiceTabular}{ccccc}[vlines]
        \Hline
        \diagbox{X}{Y} & 1   &  2   &  3 & 4\\
        \Hline
        1 & $0$& $\frac{1}{12}$& $\frac{1}{12}$& $\frac{1}{12}$\\
        \Hline
        2 & $\frac{1}{12}$& $0$& $\frac{1}{12}$& $\frac{1}{12}$\\
        \Hline
        3 & $\frac{1}{12}$& $\frac{1}{12}$& $0$& $\frac{1}{12}$\\
        \Hline
        4 & $\frac{1}{12}$& $\frac{1}{12}$& $\frac{1}{12}$& $0$\\
        \Hline
    \end{NiceTabular}
\end{center}

$P(X \le Y)$ is again the sum of the upper triangle of the pmf table. In this case however the sum is now $6\times\frac{1}{12}=0.5.$

\section*{6}
%6.30ac
\subsection*{a}
\begin{align*}
    \int_0^\infty \int_x^\infty ce^{-2y} dydx &= 1\\
    \int_0^\infty -\frac{c}{2}e^{-2y} \Bigr |_x^\infty dx &= 1\\
    \int_0^\infty 0 + \frac{c}{2}e^{-2x}dx &= 1\\
    \int_0^\infty \frac{c}{2}e^{-2x}dx &= 1\\
    -\frac{c}{4}e^{-2x} \Bigr |_0^\infty &= 1\\
    \frac{c}{4} &= 1\\
    c &= 4
\end{align*}

\subsection*{c}
Our bounds are now $0<x<y<2x<\infty.$
\begin{align*}
    P(Y<2X) &= \int_0^\infty \int_x^{2x} 4e^{-2y} dydx\\
    &= \int_0^\infty -\frac{4}{2}e^{-2y} \Bigr |_x^{2x} dx\\
    &= \int_0^\infty -\frac{4}{2}e^{-4x} + \frac{4}{2}e^{-2x} dx\\
    &= \frac{4}{8}e^{-4x} - \frac{4}{4}e^{-2x} \Bigr |_0^\infty\\
    &= 0 - (\frac{1}{2} - 1)\\
    &= \frac{1}{2}
\end{align*}

\section*{7}
\subsection*{a}
Our bounds are $0<x<y<\infty.$ Thus:
\begin{align*}
    P(X<Y) &= \int_0^\infty\int_0^y 0.5e^{-x}e^{-y/2}dxdy\\
    &= \int_0^\infty -0.5e^{-x}e^{-y/2}\Bigr |_0^y dy\\
    &= \int_0^\infty -\frac{1}{2}e^{-3y/2} +  \frac{1}{2}e^{-y/2}dy\\
    &= \frac{1}{3}e^{-3y/2} - 1e^{-y/2} \Bigr |0^\infty\\
    &= 0 - (\frac{1}{3}-1)\\
    &= \frac{2}{3}
\end{align*}

\subsection*{c}
$P(X \text{is within 2 of } Y) = 1 - P(X \text{is not within 2 of } Y).$ $P(X \text{is not within 2 of } Y)$ is true when either $0<x<x+2\le y<\infty$ or $0<y<y+2\le x<\infty.$ Thus:
\begin{align*}
    P(X \text{ is not within 2 of } Y) &=  P(Y \text{ is at least 2 above } X) + P(X \text{ is at least 2 above } Y)\\
    &= \int_0^\infty\int_{x+2}^\infty 0.5e^{-x}e^{-y/2}dydx +  \int_0^\infty\int_{y+2}^\infty 0.5e^{-x}e^{-y/2}dxdy
\end{align*}
Hence:
\begin{align*}
    P(X \text{ is within 2 of } Y) &= 1 - P(X \text{ is not within 2 of } Y)\\
    &= 1- \int_0^\infty\int_{x+2}^\infty 0.5e^{-x}e^{-y/2}dydx -  \int_0^\infty\int_{y+2}^\infty 0.5e^{-x}e^{-y/2}dxdy
\end{align*}

\section*{8}
This is a right triangle with base $1$ and height $1$. Thus, its area is $\frac{1}{2}.$

By definition of uniform distribution in two dimensions, given $S$ is the bounded area, i.e. the triangle:
\[f(x,y)=\frac{1}{\text{Area}(S)}=2\]
To be more formal with bound we would thus have 
\begin{equation*}
    f(x,y)=\begin{cases}
        2,\; x\ge0,x\le1,y\le x\\
        0,\; o.w.
    \end{cases}
\end{equation*}
using the fact that the triangle with points $(0,0), (0,1), (1,1)$ is bounded by the lines $x=0,x=1,y=x.$
\end{document}