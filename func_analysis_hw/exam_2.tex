\documentclass[10pt]{article}
\setlength{\textwidth}{6.3in}
\setlength{\textheight}{9in}
\setlength{\oddsidemargin}{0in}
\setlength{\evensidemargin}{0in}
\setlength{\topmargin}{-.5in}
%\parindent=0in
\linespread{1.3}
\usepackage{ mathrsfs }
\usepackage{amsthm}
\usepackage{ amssymb }
\usepackage{graphicx}
\newtheorem{theorem}{Theorem}[section]
\newtheorem{lemma}[]{Lemma}
\newtheorem{definition}[]{Definition}

\usepackage{amsmath}
\usepackage{amsfonts}
\usepackage{fancyhdr}
\usepackage{nicematrix}
\usepackage{mathtools}
\usepackage{enumerate}
\pagestyle{fancy}
\headheight = 14.5pt
\lhead{Functional Analysis Exam 2, Thomas Zeng }
\rhead{Math 331, Spring 2023}
\cfoot{\thepage}


\begin{document}

\section*{Problem 1}

\subsection*{(a)}

\begin{proof}
    We first show that $A^{-1}$ is closed under addition. Namely, given arbitrary $y_1, y_2 \in X$ s.t. $A^{-1}y_1 = x_1$ and $A^{-1}y_2 = x_1,$ clearly we have that
    \[A^{-1}y_1 + A^{-1}y_2 = x_1 + y_2.\]
    We similarly have that
    \begin{alignat*}{2}
        &Ax_1 + Ax_2 = y_1 + y_2 \qquad&&\text{as $A$ is bijection}\\ 
        \Rightarrow& A(x_1 + x_2) = y_1 + y_2 &&\text{linearity of $A$}\\
        \Rightarrow& x_1 + x_2 = A^{-1}(y_1 + y_2).
    \end{alignat*}
    Therefore,
    \[A^{-1}y_1 + A^{-1}y_2 = A^{-1}(y_1+y_2)\]
    as desired.

    We next claim that $A^{-1}$ is closed under multiplication. Namely, given arbitrary $\alpha\in\mathbb{C}$ and $y\in X,$ s.t.
    \[A^{-1}(\alpha y)=x,\]
    it follows that
    \begin{alignat*}{2}
        &\alpha y = Ax\qquad&&\text{as $A$ is bijection}\\
        \Rightarrow& y = \frac{1}{\alpha}Ax\\
        \Rightarrow& y = A \frac{x}{\alpha} &&\text{linearity of $A$}\\
        \Rightarrow& A^{-1}y = \frac{x}{\alpha} &&\text{as $A$ is bijection}\\
        \Rightarrow& \alpha A^{-1}y = x.
    \end{alignat*}
    In other words, we have that
    \[A^{-1}(\alpha y) = \alpha A^{-1}y,\]
    thus proving closure under multiplication.
\end{proof}


\subsection*{(b)}

\begin{proof}
    We first claim that
    \begin{equation}\label{eq:infb}
        \inf_{x\in X,\Vert x\Vert=1}\Vert Ax\Vert >0.
    \end{equation}
    Namely, as $A$ is a bounded linear operator between Banach spaces (namely $X$ to $X$) that is bijective and therefore surjective, it thus maps open sets to open sets by the open mapping theorem. Therefore, $A$ must map $B_{0.5}(0)$ to some open set $A(B_{0.5}(0))$ in $X.$ As $A(0)=0$ \emph{a fortiori} from the linear property of linear operators, it follows that there exists some $\delta > 0$ s.t. $B_\delta(0)\subseteq A(B_{0.5}(0)).$ Then by the injective nature of $A,$ it follows $Ax \in B_\delta(0)$ implies that $\Vert x\Vert < 0.5,$ and thus proving \eqref{eq:infb}.

    Now given arbitrary $u\in X$ where $\Vert u\Vert = 1$ and $A^{-1}u = y,$ it follows that
    \begin{alignat*}{2}
        &Ay = u &&\text{as $A$ is bijection}\\
        \Rightarrow& \left\Vert A \frac{y}{\Vert y\Vert}\right\Vert = \left\Vert \frac{u}{\Vert y\Vert}\right\Vert &&\text{linearity of $A$}\\
        \Rightarrow& \left\Vert \frac{u}{\Vert y\Vert}\right\Vert \ge \inf_{x\in X,\Vert x\Vert=1}\Vert Ax\Vert\\
        \Rightarrow& \Vert u\Vert \ge \Vert y\Vert\inf_{x\in X,\Vert x\Vert=1}\Vert Ax\Vert &&\text{pos. homo. of norm}\\
        \Rightarrow& 1 \ge \Vert y\Vert\inf_{x\in X,\Vert x\Vert=1}\Vert Ax\Vert &&\text{as $u$ is unit}\\
        \Rightarrow& \Vert y\Vert \le \frac{1}{\inf_{x\in X,\Vert x\Vert=1}\Vert Ax\Vert} \qquad&&\text{well--defined by \eqref{eq:infb}}\\
        \Rightarrow& \left\Vert A^{-1}u\right\Vert \le \frac{1}{\inf_{x\in X,\Vert x\Vert=1}\Vert Ax\Vert}.
    \end{alignat*}
    As $u$ was arbitrary, this implies that
    \begin{equation} \label{eq:ai}
        \Vert A^{-1} \Vert \le \frac{1}{\inf_{x\in X,\Vert x\Vert=1}\Vert Ax\Vert}.
    \end{equation}
    This shows that $A^{-1}$ is bounded. To show that \eqref{eq:ai} is actually an equality, it suffices to show that we can find $u$ s.t. $\Vert A^{-1}u \Vert$ is arbitrary close to the right--hand side of the equation.
    Namely, for any $\epsilon>0,$ by the fact that infimum is the greatest lower bound (and some algebra I'm too lazy to do), we can choose unit vector $x_1 \in X$ s.t.
    \begin{equation}\label{eq:aar}
        \left|\left\Vert Ax_1\right\Vert^{-1} - \left(\inf \Vert Ax\Vert\right)^{-1}\right| < \epsilon.
    \end{equation}
    Now, as we have
    \begin{alignat*}{2}
        \left \Vert A^{-1}\frac{Ax_1}{\Vert Ax_1\Vert} \right \Vert &= \left \Vert \frac{x_1}{\Vert Ax_1\Vert}\right\Vert\\
        &= \frac{1}{\Vert Ax_1\Vert},
    \end{alignat*}
    it thus follows with \eqref{eq:aar} that
    \[\left|\left \Vert A^{-1}\frac{Ax_1}{\Vert Ax_1\Vert} \right \Vert - \left(\inf \Vert Ax\Vert\right)^{-1}\right| < \epsilon.\]
    As $Ax_1/\Vert Ax_1\Vert$ is a unit vector, this thus further implies that
    \[\left|\left \Vert A^{-1} \right \Vert - \left(\inf \Vert Ax\Vert\right)^{-1}\right| < \epsilon.\]
    And hence \eqref{eq:ai} is an equality as desired.
\end{proof}

\subsection*{(c)}

We have that
\begin{alignat*}{2}
    & \left \Vert A^{-1} \right\Vert^{-1} = \inf_{x\in X \Vert x\Vert = 1}\left\Vert Ax\right\Vert\\
    \Rightarrow& \left \Vert A^{-1} \right\Vert^{-1} \le \sup_{x\in X \Vert x\Vert = 1}\left\Vert Ax\right\Vert\\
    \Rightarrow& \left\Vert A^{-1} \right\Vert^{-1} \le \Vert A\Vert\\
    \Rightarrow&\Vert A\Vert^{-1} \le \left\Vert A^{-1} \right\Vert.
\end{alignat*}

Now, as an example to show that this is not an equality, consider in $\mathbb{R}^2$ the linear operator
\[A = \begin{bmatrix}
    2 & 0\\
    0 & 1
\end{bmatrix}.\]
As $A$ is an invertible matrix in finite dimensions, it is clearly a bijective bounded linear operator. We have $\Vert A\Vert = 2$ if we consider $u = (1,0)$ and $\Vert A^{-1}\Vert = (\inf \Vert Ax\Vert)^{-1}= 1$ if we consider $u = (0,1).$ We thus have that
\[\Vert A\Vert^{-1} = \frac{1}{2}< 1 = \left \Vert A^{-1}\right\Vert.\]

\section*{Problem 2}

\subsection*{(a)}

\begin{proof}
    It suffices to prove the contrapositive. Assume $\Vert A \Vert$ is unbounded. Then there exists sequence of unit vectors $(x_n)$ s.t. the corresponding sequence $(Ax_n)$ has the property that for all $n > 1$ we have
    \[\Vert Ax_n\Vert - \Vert Ax_{n-1}\Vert \ge 1.\]
    It then follows by the reverse--triangle inequality, that for any $m,n\in\mathbb{N},$ we have that
    \[\Vert A_m - A_n \Vert \ge | \Vert A_m\Vert - \Vert A_n\Vert | \ge 1.\]
    Thus, there is no Cauchy subsequence and hence no convergent subsequence in $(Ax_n)$ as convergence implies Cauchy. As $(x_n)$ is bounded, therefore $A$ is not compact.
\end{proof}

\subsection*{(b)}

\begin{proof}
    It suffices to show that compact operators are closed under addition and multiplication.

    We first show closure under multiplication. Namely, let $(f_n)$ be any arbitrary bounded--sequence in $X.$ It follows by definition, that $(Af_n)$ contains some convergent subsequences. It thus follows by the algebraic limit theorem that $(cAf_n)$ also contains some convergent subsequences for any $c\in\mathbb{C}$. Thus, $cA$ is a compact operator.

    We next show closure under addition. Let $A, B$ be two compact operators. Let $(f_n)$ be some bounded sequence. Then $(Bf_n)$ must contain some convergent subsequence $(Bf_{n_x}).$ This convergent subsequence must correspond to some subsequence $(f_{n_x})$ in $(f_n).$ As any subsequence of a bounded sequence is bounded, it thus follows that $(f_{n_x})$ is bounded and hence has a convergent subsequence in $(Af_{n_x}).$ It follows by the algebraic limit theorem that $(Af_{n_x} + Bf_{n_x})$ must contain a convergent subsequence. And thus $(Af_{n} + Bf_{n})$ must also. Thus, $A+B$ is compact.
\end{proof}

\subsection*{(c)}

\begin{proof}
    We first note that $\mathscr{L}(X)$ is a normed-algebra, and is thus associative.

    We now show $AB$ is compact. Namely, let $(f_n)$ be any bounded sequence in $X.$ It follows that $(Bf_n)$ is bounded as bounded operators map bounded sets to bounded sets. Therefore, $(ABf_n)$ contains a convergent subsequence since $A$ is compact. Hence, $AB$ is compact.

    We next show that $BA$ is compact. Again, let $(f_n)$ be some bounded sequence in $X.$ By compactness, $(Af_n)$ contains some convergent subsequence. Therefore, $(BAf_n)$ must contain a convergent subsequence by the continuity of bounded linear operators.
\end{proof}

\section*{Problem 3}

\subsection*{(a)}
\begin{proof}
    We first show that $A$ is linear. Namely, given arbitrary $f_1, f_2 \in \mathcal{L}^2_{\text{cont}}(0,1)$ and $
    \alpha\in\mathbb{C}$ we have that:
    \begin{alignat}{2}
        A(\alpha f_1 + f_2): s &\mapsto \int_{0}^{s}k(s,t)[\alpha f_1 + f_2](t)dt \nonumber\\
        &\mapsto \int_{0}^{s}k(s,t)[\alpha f_1(t) + f_2(t)]dt \nonumber\\
        &\mapsto \alpha \int_{0}^{s}k(s,t)f_1(t)dt + \int_{0}^{s}k(s,t)f_2(t)dt \quad&&\text{by alg. int. theorems.}\label{eq:ait}
    \end{alignat}
    Observe that \eqref{eq:ait} is exactly the function defined by $\alpha Af_1 + Af_2.$ Thus, we have shown linearity.

    We next claim that $A$ is bounded.
\end{proof}

\end{document}