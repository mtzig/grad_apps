\documentclass[10pt]{article}
\setlength{\textwidth}{6.3in}
\setlength{\textheight}{9in}
\setlength{\oddsidemargin}{0in}
\setlength{\evensidemargin}{0in}
\setlength{\topmargin}{-.5in}
%\parindent=0in
\linespread{1.3}
\usepackage{ mathrsfs }
\usepackage{amsthm}
\usepackage{ amssymb }
\usepackage{graphicx}
\newtheorem{theorem}[]{Theorem}
\newtheorem{lemma}[]{Lemma}
\newtheorem{definition}[]{Definition}

\usepackage{amsmath}
\usepackage{amsfonts}
\usepackage{fancyhdr}
\usepackage{nicematrix}
\usepackage{mathtools}

\pagestyle{fancy}
\headheight = 14.5pt
\lhead{Functional Analysis Exam 1 Review, Thomas Zeng }
\rhead{Math 331, Spring 2023}
\cfoot{\thepage}


\begin{document}

\section{Defintions}

\begin{definition}[Norm]
    A nonnegative function $\Vert.\Vert:X\to[0,\infty)$ s.t.
    \begin{itemize}
        \item $\Vert f\Vert>0$ for $f\in X\backslash\{0\}$ \textbf{Positive definiteness}
        \item $\Vert\alpha f\Vert = |\alpha|\Vert f\Vert$ for all $\alpha\in\mathbb{C},$ $f\in X$ \textbf{positive homogeneity}
        \item $\Vert f+g\Vert \le \Vert f\Vert +\Vert g\Vert$ for all $f,g\in X$ \textbf{triangle inequality}.
    \end{itemize}
    If positive definiteness is dropped form the requirements, one calls $\Vert.\Vert$ a \textbf{seminorm}.
\end{definition}

\begin{definition}[Cauchy Sequence]
    A sequence $f_n$ is Cauchy if for every $\epsilon>0,$ there is some $N$ s.t. $\Vert f_n-f_m\Vert < \epsilon$ for all $n,m\ge N.$
\end{definition}

\begin{definition}[Completeness]
    A normed space is called \textbf{complete} if every Cauchy sequence has a limit.
\end{definition}

\begin{definition}[Banach Space]
    A complete normed vector space (over $\mathbb{C}$).
\end{definition}

\begin{definition}[Open Ball]
    The open ball of radius $\epsilon$ centered around $f\in X$ is defined as $B_\epsilon(f)\coloneqq\{g\in X|\Vert f-g\Vert<\epsilon\}\subseteq X.$
\end{definition}

\begin{definition}[Dense Set]
    A subset $D \subseteq X$ is dense in $X$ if its closure is $X$ i.e. any neighborhood of any point $x\in X$ must intersect $D.$
\end{definition}

\begin{definition}[Closed Subspace]
    A subset $C \subseteq X$ that contains all its limit points.
\end{definition}

\begin{definition}[Span of set of Vectors]
    The set of all finite linear combinations of a set of vectors $\{ u_n\}_{n\in\mathcal{N}}\subset X$ is called the \textbf{span} of $\{u_n\}_{n\in\mathcal{N}}.$
\end{definition}

\begin{definition}[Linearly independent set of Vectors]
    A set of vectors $\{ u_n\}_{n\in\mathcal{N}}\subset X$ is called \textbf{linearly independent} if every finite subset is.
\end{definition}

\begin{definition}[Hamel Basis]
    A maximal set of linearly independent vectors.
\end{definition}

\begin{definition}[Schauder basis]
    A countable sequence of vectors $(u_n)_{n=1}^N$ from $X$ is a \textbf{Schauder basis} if every element $f\in X$ can be uniquely written as a countable linear combination of the basis elements:
    \[f=\sum_{n=1}^N\alpha_n u_n,\qquad \alpha_n\in\mathbb{C}\]
    where $\alpha_n$ is uniquely determined by $f$.
\end{definition}

\begin{definition}[Total Set]
    A set whose span is dense is called \textbf{total}.
\end{definition}

\begin{definition}[Seperable]
    A normed vector space containing a countable dense set is called \textbf{separable}.
\end{definition}

\begin{definition}[Sesquilinear Form]
    Suppose $\mathfrak{H}$ is a vector space. A map $\langle,\rangle:\mathfrak{H}\times\mathfrak{H}\to\mathbb{C}$ is called a \textbf{sesquilinear form} if is conjugate linear in the first argument and linear in the second.
\end{definition}

\begin{definition}[Hermitian form]
    A symmetric sesquilinear form is called a Hermitian form.
\end{definition}

\begin{definition}[Inner Product]
    A positive definite Hermitian form is called an inner product.
\end{definition}

\begin{definition}[Hilbert Space]
    The pair $(\mathfrak{H},\langle ,\rangle)$ where $\mathfrak{H}$ is a vector space and $\langle ,\rangle$ 
    an inner product, is called an \textbf{inner product space}. If $\mathfrak{H}$ is complete w.r.t the norm
    \[\Vert f\Vert \coloneqq \sqrt{\langle f,f\rangle},\]
    it is called a \textbf{Hilbert space}.
\end{definition}

\begin{definition}[Orthogonal Vectors]
    Two vectors $f,g\in\mathfrak{H}$ are called \textbf{orthogonal} if $\langle f,g\rangle =0.$
\end{definition}

\begin{definition}[Projection of Vector]
    Suppose $u$ is a unit vector. Then the projection of $f$ in the direction of $u$ is given by
    \[f_\Vert \coloneqq \langle u,f\rangle u,\]
    and
    \[f_\perp \coloneqq f-\langle u,f\rangle u.\]
\end{definition}

\section{Theorems}

\begin{theorem}[Hölder's Inequality]
    For any $p,q$ s.t. $1/p +1/q = 1,$ we have that
    \[\Vert ab\Vert_1 \le \Vert a\Vert_p\Vert b\Vert_q.\]
\end{theorem}

\begin{theorem}
    Let $I\subset \mathbb{R}$ be a compact interval, then the continuous functions $C(I)$ with the maximum norm     form a Banach space. 
\end{theorem}

\begin{theorem}[Weierstraß Approximation]
    Let $I\subset\mathbb{R}$ be a compact interval. Then the set of polynomials is dense in $C(I).$
\end{theorem}

\begin{theorem}[Pythagorean Theorem]
    If $f$ and $g$ are orthogonal, then
    \[\Vert f+g\Vert^2 =\Vert f\Vert^2 +\Vert g\Vert^2.\]
\end{theorem}

\begin{theorem}[Cauchy-Schwartz]
    Let $\mathfrak{H}_0$ be an inner product space. Then for every $f,g\in\mathfrak{H}_0$ we have
    \[|\langle f,g\rangle|\le \Vert f\Vert\Vert g\Vert\]
    with equality iff $f$ and $g$ are parallel.
\end{theorem}

\begin{theorem}[Jordan-von Neumann]
    A norm is associated with a scalar product iff the \textbf{parallelogram law}
    \[\Vert f+g\Vert^2 + \Vert f-g\Vert^2 = 2\Vert f\Vert^2 + 2\Vert g\Vert^2\]
    holds.

    In this case, the scalar product can be recovered from its norm by virtue of the \textbf{polarization identity}
    \[\langle f,g\rangle = \frac{1}{4}(\Vert f+g\Vert^2 -\Vert f-g\Vert^2 + i\Vert f-ig\Vert^2-i\Vert f+ig\Vert^2).\]
\end{theorem}

\end{document}