\documentclass[10pt]{article}
\setlength{\textwidth}{6.3in}
\setlength{\textheight}{9in}
\setlength{\oddsidemargin}{0in}
\setlength{\evensidemargin}{0in}
\setlength{\topmargin}{-.5in}
%\parindent=0in
\linespread{1.3}
\usepackage{ mathrsfs }
\usepackage{amsthm}
\usepackage{ amssymb }
\usepackage{graphicx}
\newtheorem{theorem}{Theorem}[section]
\newtheorem{lemma}[]{Lemma}
\newtheorem{definition}[]{Definition}

\usepackage{amsmath}
\usepackage{amsfonts}
\usepackage{fancyhdr}
\usepackage{nicematrix}

\pagestyle{fancy}
\headheight = 14.5pt
\lhead{Real Analysis PS6, Thomas Zeng }
\rhead{Math 321, Winter 2023}
\cfoot{\thepage}

\begin{document}

\section*{5.2.3}

\subsection*{a}

% prove deriv of h(x)=1/x

Let $c$ be in the domain of $h$. We thus have by definition of derivative that
\begin{equation} \label{eq:ddef}
    h'(c) = \lim_{x\to c}\frac{h(x)-h(c)}{x-c}. 
\end{equation}
Now from a past example in the textbook, we know that for any $x,c\neq 0,$ the following identity holds:
\begin{equation} \label{eq:dident}
    1/x-1/c = \frac{c-x}{xc}.
\end{equation}
Therefore, we have
\begin{alignat}{2}
    \frac{h(x)-h(c)}{x-c} &= \frac{1/x-1/c}{x-c} &&\text{by def. of }h\nonumber\\
    &= \frac{(c-x)/(xc)}{x-c} &&\text{by \eqref{eq:dident}}\nonumber\\
    &=-\frac{(x-c)/(xc)}{x-c} \qquad&&\text{algebra}\nonumber\\
    &= -1/(xc) \label{eq:simp}.
\end{alignat}
Therefore we have that
\begin{alignat*}{2}
    h'(c) &= \lim_{x\to c}\frac{h(x)-h(c)}{x-c} \qquad&&\text{by \eqref{eq:ddef}}\\
    &= \lim_{x\to c} -\frac{1}{xc} &&\text{by \eqref{eq:simp}}\\
    &= -c^{-2} &&\text{by Alegebraic Theorem for Functional Limits}.
\end{alignat*}
\subsection*{b}
    % use part a and chain rule to show 5.2.4iv
    \begin{proof}
        Consider $h(x)=1/x.$ We thus have by algebra that
        \begin{equation} \label{eq:funci}
            f/g = f[h\circ g].
        \end{equation}
        Therefore, provided $g(c)\neq 0$ we thus have
        \begin{alignat*}{2}
            (f/g)'(c) &= (f[h\circ g])'(c) &&\text{by \eqref{eq:funci}}\\
            &=f'(c)(h\circ g)(c) + f(c) (h\circ g)'(c) &&\text{by product rule}\\
            &= f'(c)/g(c) + f(c)(h\circ g)'(c) &&\text{as }h\circ g(c)=1/g(c)\\
            &= f'(c)/g(c) + f(c)[-g(c)^{-2}]g'(c) \qquad&&\text{by previous subquestion}\\
            &= \frac{f'(c)g(c)}{g(c)^2} - \frac{f(c)g'(c)}{g(c)^2}\  &&\text{algebra}\\
            &= \frac{g(c)f'(c)-f(c)g'(c)}{[g(c)]^2}.
        \end{alignat*}
    \end{proof}

\subsection*{c}
\begin{proof}
    First we rewrite the difference quotient as
    \begin{alignat}{2}
        \frac{(f/g)(x)-(f/g)(c)}{x-c} &= \frac{f(x)/g(x)-f(c)/g(c)}{x-c} &&\text{as }(f/g)(x)=f(x)/g(x)\nonumber\\
        &= \frac{f(x)/g(x)-f(c)/g(x)+f(c)/g(x)-f(c)/g(c)}{x-c} \qquad&&\text{algebra}\nonumber\\
        &=\frac{1}{g(x)}\frac{f(x)-f(c)}{x-c}-f(c)\frac{1/g(c)-1/g(x)}{x-c} &&\text{algebra}\nonumber\\
        &=\frac{1}{g(x)}\left [ \frac{f(x)-f(c)}{x-c}\right ] - \frac{f(c)}{g(c)g(x)}\left [ \frac{g(x)-g(c)}{x-c}\right ]. &&\text{by \eqref{eq:dident}} \label{eq:difq}
    \end{alignat}
    We now note that as $f,g$ are differentiable and thus continuous, it holds that $\lim_{x\to c}f(x)=f(c)$ and $\lim_{x\to c}g(x)=g(c).$ It thus follows that
    \begin{alignat*}{2}
        (f/g)'(c)&=\lim_{x\to c}  \frac{(f/g)(x)-(f/g)(c)}{x-c} &&\text{by def. of derivative}\\
        &= \lim_{x\to c}\frac{1}{g(x)}\left [ \frac{f(x)-f(c)}{x-c}\right ] - \frac{f(c)}{g(c)g(x)}\left [ \frac{g(x)-g(c)}{x-c}\right ] \qquad&&\text{by \eqref{eq:difq}}\\
        &= \frac{1}{g(c)}\lim_{x\to c}\left [\frac{f(x)-f(c)}{x-c} \right ] - \frac{f(c)}{g(c)^2}\lim_{x\to c}\left [\frac{g(x)-g(c)}{x-c}\right ] &&\text{by functional limit ALT}\\
        &=\frac{f'(c)}{g(c)}-\frac{f(c)g'(c)}{g(c)^2} &&\text{by def. of derivative}\\
        &=\frac{g(c)f'(c)-f(c)g'(c)}{[g(c)]^2}. &&\text{algebra}
    \end{alignat*}
    % \begin{alignat*}{2}
    %     (f/g)'(c)&=
    % \end{alignat*}
\end{proof}

\section*{5.2.7}
\subsection*{a}
Consider $a=1.1.$ 
\begin{lemma} \label{lem:1}
    Given $a>1,$ then $g_a$ is differentiable.
\end{lemma}

\begin{proof}
    At any point besides $x=0,$ we have that $g_a$ is differentiable for any $a$ since it is the product and composition of differentiable functions (at any point besides $0$). Thus, we only need to consider the value of $a$ s.t. $g_a'(0)$ exists or in other words that
    \begin{equation} \label{eq:dconv}
        \lim_{x\to 0}\frac{g_a(x)-g_a(0)}{x-0}=\lim_{x\to 0}\frac{x^a\sin1/x}{x}
    \end{equation}
    converges. As $|\sin 1/x|\le 1,$ we note that  for all $x\in\mathbb{R}$ we have 
    \begin{equation}\label{eq:squeezeone}
        -x^a/x\le (x^a\sin1/x)/x \le x^a/x.
    \end{equation}
    Furthermore, we note that for $a>1$ we have that 
    \begin{equation}\label{eq:squeeze}
        \lim_{x\to 0}x^a/x=\lim_{x\to 0}-x^a/x=0.
    \end{equation}
    Thus, given $a>1,$ with \eqref{eq:squeezeone} and \eqref{eq:squeeze} we use the Squeeze Theorem (book exercise 4.2.11) to claim that \eqref{eq:dconv} converges to $0$ as well. Therfore $g_a'(0)$ exists when $a>1.$
\end{proof}

\begin{lemma} \label{lem:2}
    Given $1<a<2,$ then $g_a'$ is unbounded on $[0,1]$.
\end{lemma}

\begin{proof}
    We first go back to calculus, to claim that for any $a$ we have that
    \begin{equation} \label{eq:gendiv}
        g_a'(x) = \underbrace{ax^{a-1}\sin 1/x}_{(i)} - \overbrace{x^{a-2} \cos1/x}^{(ii)}
    \end{equation}
    where we use the product rule and chain rule. Now for any $b<0$ we have that $x^b$ is unbounded on the interval $[0,1]$. And similarly, if $b>0,$ then $x^b$ is bounded on the interval $[0,1]$ (namely by $0$ and $1$).
    
    Therefore, by above, when $1<a < 2,$ we have that ($i$) is bounded and $(ii)$ is not and therefore \eqref{eq:gendiv} is unbounded.
\end{proof}

Thus by lemma \ref{lem:1} and \ref{lem:2}, we have that $a=1.1$ works.

\subsection*{b}

Consider $a=3.$ Following the same format as \eqref{eq:gendiv}, we would thus have
\begin{equation} \label{eq:gendiv3}
    g_3'(x) = \underbrace{3x^{2}\sin 1/x}_{(i)} - \overbrace{x \cos1/x}^{(ii)}
\end{equation}
By lemma \ref{lem:1} we would have that ($i$) in \eqref{eq:gendiv3} is differentiable (as it is the composition of differentiable functions). However. ($ii$) in \eqref{eq:gendiv3} is not differentiable -- namely at $0$ (by similar proof as in book for $x\sin 1/x$). Therefore, $g_3'$ is not differentiable -- as otherwise we could use the Alegebraic Differentiability Theorem to show that ($ii$) is differentiable.

Now we note that ($i$) is continuous since it is differentiable. For ($ii$) we can use proof similar as in the book for $x\sin 1/x$ to show that it is continuous. Therefore, by the Algebraic Continuity Theorems, \eqref{eq:gendiv3} is continuous.

\section*{5.3.0}

% if derivative is always greater than 0, then function is strictly increasing

% consider points wlog assume y greater than x

\begin{proof}
    Given $x,y\in A$ where $x<y$. The restriction of $g$ to the interval $[x,y]$ is necessarily continuous on $[x,y]$ as $g$ is differentiable on $A\supseteq [x,y].$ By similar logic, it is differentiable on $(x,y).$ Therefore, by the MVT, there exists $c\in (x,y)$ s.t.
    \begin{equation*}
        g'(c) = \frac{g(y)-g(x)}{y-x}.
    \end{equation*}
    Therefore, by our assumptions we have that
    \begin{equation} \label{eq:gz}
        \frac{g(y)-g(x)}{y-x} > 0.
    \end{equation}
    As we stated that $x<y$, it follows that $y-x > 0.$ Therefore, in order for \eqref{eq:gz} to hold, it must be that $g(y)>g(x).$ Therefore $g$ is strictly increasing.
\end{proof}

\section*{5.3.1}

\subsection*{a} 

\begin{proof}
    As $A=[a,b]$ is closed and bounded, it is therefore compact. As $f'$ is continuous, therefore, by the Extreme Value Theorem, there exists $K$ s.t. 
    \begin{equation} \label{eq:evt}
        |f'(x)|\le K
    \end{equation}
    for all $x \in A.$

    Now as $f$ is differentiable on $[a,b]$ it is necessarily continuous on $[a,b]$ (as all differentiable functions are continuous) and differentiable on $(a,b).$ Now let $x,y\in A.$ By the MVT, there exists $c\in A$ s.t.
    \begin{equation*}
        \frac{f(y)-f(x)}{y-x}=f'(c).
    \end{equation*}

    By \eqref{eq:evt}, we know that $|f'(c)|\le K.$ Therefore, $|\frac{f(y)-f(x)}{y-x}|\le K.$ As $x,y$ were arbitrary, therefore $f$ is Lipschitz.
\end{proof}

% \begin{lemma} \label{lem:cs}
%     Let $f:[a,b]\to\mathbb{R},$ not be Lipschitz. Now consider $c\in (a,b),$ it follows that either the restriction $f|_{[a,c]}$ or $f|_{[c,b]}$ is not Lipschitz.
% \end{lemma}

% \begin{proof}
%     We prove this by contradiction. That is assume both $f|_{[a,c]}$ and $f|_{[c,b]}$ are Lipschitz. Therefore there exists $M_1$ and $M_2$ that serve as an upper bound on the ``slopes'' of $f|_{[a,c]}$ and $f|_{[c,b]}$ respectively. Now let $M = \max\{M_1,M_2\}.$ Now for any $x,y\in [a,c]$ or $x,y\in[c,b],$ it follows by our assumption that
%     \begin{equation*}
%         \left | \frac{f(x)-f(y)}{x-y}\right |\le M.
%     \end{equation*}
%     Next, if we have $y\in [a,c]$ and $x\in[c,b]$ where $x\neq y,$ we have that
%     \begin{alignat*}{2}
%         \left | \frac{f(x)-f(y)}{x-y}\right | &=  \left | \frac{f(x)-f(c)+f(c)+f(y)}{x-y}\right | &&\text{algebra}\\
%         &\le \left | \frac{f(x)-f(c)}{x-y} \right | + \left | \frac{f(c)-f(y)}{x-y} \right | \qquad&&\text{triangle inequality}\\
%         &\le \left | \frac{f(x)-f(c)}{x-c} \right | + \left | \frac{f(c)-f(y)}{c-y} \right | &&\text{as }y\le c \le x\\
%         &\le M + M &&\text{by our def. of }M\\
%         &= 2M.
%     \end{alignat*}
%     Therefore, we have $f$ is Lipschitz, a contradiction.
% \end{proof}

% \noindent
% \textbf{Now we prove this problem.}
% \begin{proof}
%     First we assume that $f$ is not Lipschitz.
%     We now construct a sequence $(c_n)$ as follows. As $f$ is not Lipschitz, we can find some $x,y\in A$ s.t.
%     \begin{equation*}
%         \left | \frac{f(x)-f(y)}{x-y}\right |> 1.
%     \end{equation*}
%     As $f$ is differentiable on $A,$ it follows that $f$ is differentiable and continuous on $[x,y].$ Therefore, by the MVT, there exists some point $c\in [x,y]$ where $|f'(c)|>1.$ Let $c_1 = c.$ By Lemma \ref{lem:cs} we are guaranteed to be able to continue this process. That is $f$ is not Lipschitz on either $[x,c]$ or $[c,y]$ and thus choosing the non-Lipschitz interval we can again use MVT to guarantee a $c_2$ in that interval where $|f'(c_2)|>2.$ We can repeat this process to find $c_n$ where $|f'(c_n)|>n$ in general.

%     As $(c_n)$ is in $A$ -- a closed and bounded and therefore compact set -- thus by definition, there exists a subsequence $(c_{n_k})$ that converges to some value $k.$ However, $(f'(c_{n_k}))$ does not converge to $f'(k)$ as by our construction, $(f'(c_n))$ is unbounded, therefore the subsequence $(f'(c_{n_k}))$ is divergent. Therefore, by the Criterion for Discontinuity, $f'$ is discontinuous -- a contradiction. Therefore, $f$ is Lipschitz.
% \end{proof}

\section*{5.3.5}
\subsection*{a}

\begin{proof}
    By Algebraic Continuity Theorems and Algebraic Differentiability Theorems, we have that $h$ is continuous on $[a,b]$ and differentiable on $(a,b).$ Therefore, by the MVT, there exists $c\in(a,b)$ s.t.
    \begin{alignat}{2}
        h'(c) &= \frac{[f(b)-f(a)]g(b)-[g(b)-g(a)]f(b)-[f(b)-f(a)]g(a)+[g(b)-g(a)]f(a)}{b-a} \quad&&\text{def. of MVT}\nonumber\\
        &= \frac{[f(b)-f(a)][g(b)-g(a)]-[f(b)-f(a)][g(b)-g(a)]}{b-a} &&\text{distr. prop.}\nonumber\\
        &= 0. &&\text{algebra} \label{eq:z}
    \end{alignat}
    Now using the Alegebraic Differentiability Theorems, we have that in general
    \begin{equation} \label{eq:divd}
        h'(x) = [f(b)-f(a)]g'(x)-[g(b)-g(a)]f'(x).
    \end{equation}
    Thus using \eqref{eq:divd} we have
    \begin{alignat*}{2}
        h'(c) &= [f(b)-f(a)]g'(c)-[g(b)-g(a)]f'(c) &&\text{by \eqref{eq:divd}}\\
        0 &= [f(b)-f(a)]g'(c)-[g(b)-g(a)]f'(c) \qquad&&\text{by \eqref{eq:z}}\\
        [f(b)-f(a)]g'(c) &=[g(b)-g(a)]f'(c),
    \end{alignat*}
    as desired.
\end{proof}

\section*{5.2.10}

As we have shown in exercise $5.2.7$, the equation 
\begin{align*}
    g_a(x)=\begin{cases}
        x^a\sin 1/x&x\neq 0\\
        0&x=0
    \end{cases}
\end{align*}
is differentiable on $\mathbb{R}$ for any $a>1$. Therefore, $g_2(a)$ is differentiable. Similarly, $x/2$ is a linear function and thus differentiable. Thus, by the Algebraic Differentiability Theorems, $g(x)$ which is the sum of $g_2(x)$ and $x/2$ is thus differentiable.

Now, as shown in section 5.1, we have $g_2'(0)=0$. Similarly, $(x/2)'(0) = 1/2$ by calculus. Thus again, by the Algebraic Differentiability Theorems, we have $g'(0) = 0 + 1/2 = 1/2>0.$

Now consider any open interval $(i,j)$ containing $0.$

By the Archimedean principle, we can find some $N$ s.t. for any $n\ge N$ we have $0<1/n<j.$ Now choose $a \ge N$ s.t. $a \mod 4 = 3.$ Let  $x=\frac{1}{(a+2)\pi/2}$ and $y = \frac{1}{a\pi/2}.$ We thus have $0<x<y$ and $x,y\in (i,j).$ Now note by our construction, that $\sin 1/x = 1$ and $\sin 1/y = -1.$ Therefore, we have:
\begin{align*}
    g(x) &=\frac{1}{2(b+2)\pi/2} + \left ( \frac{1}{(b+2)\pi/2} \right )^2\\
    g(y) &=\frac{1}{2b\pi/2} - \left ( \frac{1}{b\pi/2} \right )^2.
\end{align*}
Now we have
\begin{align}
    \frac{1}{2b\pi/2}-\frac{1}{2(b+2)\pi/2} &= \frac{2(b+2)\pi/2-2b\pi/2}{(b+2)\pi b\pi}\nonumber\\
    &= \frac{2\pi}{(b+2)b\pi^2}\nonumber\\
    &< \frac{2\pi}{b^2\pi^2}\nonumber\\
    &< \frac{2\pi}{b^2\pi^2/32}\nonumber\\
    &= \frac{\pi/4}{b^2(\pi/2)^2}\nonumber\\
    &<\frac{1}{b^2(\pi/2)^2} \label{eq:ineq}
\end{align}
% Similarly, we know that
% \begin{equation*}
%     \frac{1}{b\pi/2} > \frac{1}{(b+2)\pi/2}
% \end{equation*}
% and therefore
% \begin{equation} \label{eq:ineq2}
%     -\left ( \frac{1}{b\pi/2} \right )^2 < \left ( \frac{1}{(b+2)\pi/2} \right )^2.
% \end{equation}
Thus, from \eqref{eq:ineq} %and \eqref{eq:ineq2}, it follows that $$
it follows that 
\begin{equation*}
    \frac{1}{2b\pi/2} - \left ( \frac{1}{b\pi/2} \right )^2 < \frac{1}{2(b+2)\pi/2}.
\end{equation*}
Thus $g(y)<g(x).$
\end{document}